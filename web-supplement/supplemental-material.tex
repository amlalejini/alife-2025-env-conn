% Options for packages loaded elsewhere
\PassOptionsToPackage{unicode}{hyperref}
\PassOptionsToPackage{hyphens}{url}
%
\documentclass[
]{book}
\usepackage{amsmath,amssymb}
\usepackage{lmodern}
\usepackage{iftex}
\ifPDFTeX
  \usepackage[T1]{fontenc}
  \usepackage[utf8]{inputenc}
  \usepackage{textcomp} % provide euro and other symbols
\else % if luatex or xetex
  \usepackage{unicode-math}
  \defaultfontfeatures{Scale=MatchLowercase}
  \defaultfontfeatures[\rmfamily]{Ligatures=TeX,Scale=1}
\fi
% Use upquote if available, for straight quotes in verbatim environments
\IfFileExists{upquote.sty}{\usepackage{upquote}}{}
\IfFileExists{microtype.sty}{% use microtype if available
  \usepackage[]{microtype}
  \UseMicrotypeSet[protrusion]{basicmath} % disable protrusion for tt fonts
}{}
\makeatletter
\@ifundefined{KOMAClassName}{% if non-KOMA class
  \IfFileExists{parskip.sty}{%
    \usepackage{parskip}
  }{% else
    \setlength{\parindent}{0pt}
    \setlength{\parskip}{6pt plus 2pt minus 1pt}}
}{% if KOMA class
  \KOMAoptions{parskip=half}}
\makeatother
\usepackage{xcolor}
\usepackage{color}
\usepackage{fancyvrb}
\newcommand{\VerbBar}{|}
\newcommand{\VERB}{\Verb[commandchars=\\\{\}]}
\DefineVerbatimEnvironment{Highlighting}{Verbatim}{commandchars=\\\{\}}
% Add ',fontsize=\small' for more characters per line
\usepackage{framed}
\definecolor{shadecolor}{RGB}{248,248,248}
\newenvironment{Shaded}{\begin{snugshade}}{\end{snugshade}}
\newcommand{\AlertTok}[1]{\textcolor[rgb]{0.94,0.16,0.16}{#1}}
\newcommand{\AnnotationTok}[1]{\textcolor[rgb]{0.56,0.35,0.01}{\textbf{\textit{#1}}}}
\newcommand{\AttributeTok}[1]{\textcolor[rgb]{0.77,0.63,0.00}{#1}}
\newcommand{\BaseNTok}[1]{\textcolor[rgb]{0.00,0.00,0.81}{#1}}
\newcommand{\BuiltInTok}[1]{#1}
\newcommand{\CharTok}[1]{\textcolor[rgb]{0.31,0.60,0.02}{#1}}
\newcommand{\CommentTok}[1]{\textcolor[rgb]{0.56,0.35,0.01}{\textit{#1}}}
\newcommand{\CommentVarTok}[1]{\textcolor[rgb]{0.56,0.35,0.01}{\textbf{\textit{#1}}}}
\newcommand{\ConstantTok}[1]{\textcolor[rgb]{0.00,0.00,0.00}{#1}}
\newcommand{\ControlFlowTok}[1]{\textcolor[rgb]{0.13,0.29,0.53}{\textbf{#1}}}
\newcommand{\DataTypeTok}[1]{\textcolor[rgb]{0.13,0.29,0.53}{#1}}
\newcommand{\DecValTok}[1]{\textcolor[rgb]{0.00,0.00,0.81}{#1}}
\newcommand{\DocumentationTok}[1]{\textcolor[rgb]{0.56,0.35,0.01}{\textbf{\textit{#1}}}}
\newcommand{\ErrorTok}[1]{\textcolor[rgb]{0.64,0.00,0.00}{\textbf{#1}}}
\newcommand{\ExtensionTok}[1]{#1}
\newcommand{\FloatTok}[1]{\textcolor[rgb]{0.00,0.00,0.81}{#1}}
\newcommand{\FunctionTok}[1]{\textcolor[rgb]{0.00,0.00,0.00}{#1}}
\newcommand{\ImportTok}[1]{#1}
\newcommand{\InformationTok}[1]{\textcolor[rgb]{0.56,0.35,0.01}{\textbf{\textit{#1}}}}
\newcommand{\KeywordTok}[1]{\textcolor[rgb]{0.13,0.29,0.53}{\textbf{#1}}}
\newcommand{\NormalTok}[1]{#1}
\newcommand{\OperatorTok}[1]{\textcolor[rgb]{0.81,0.36,0.00}{\textbf{#1}}}
\newcommand{\OtherTok}[1]{\textcolor[rgb]{0.56,0.35,0.01}{#1}}
\newcommand{\PreprocessorTok}[1]{\textcolor[rgb]{0.56,0.35,0.01}{\textit{#1}}}
\newcommand{\RegionMarkerTok}[1]{#1}
\newcommand{\SpecialCharTok}[1]{\textcolor[rgb]{0.00,0.00,0.00}{#1}}
\newcommand{\SpecialStringTok}[1]{\textcolor[rgb]{0.31,0.60,0.02}{#1}}
\newcommand{\StringTok}[1]{\textcolor[rgb]{0.31,0.60,0.02}{#1}}
\newcommand{\VariableTok}[1]{\textcolor[rgb]{0.00,0.00,0.00}{#1}}
\newcommand{\VerbatimStringTok}[1]{\textcolor[rgb]{0.31,0.60,0.02}{#1}}
\newcommand{\WarningTok}[1]{\textcolor[rgb]{0.56,0.35,0.01}{\textbf{\textit{#1}}}}
\usepackage{longtable,booktabs,array}
\usepackage{calc} % for calculating minipage widths
% Correct order of tables after \paragraph or \subparagraph
\usepackage{etoolbox}
\makeatletter
\patchcmd\longtable{\par}{\if@noskipsec\mbox{}\fi\par}{}{}
\makeatother
% Allow footnotes in longtable head/foot
\IfFileExists{footnotehyper.sty}{\usepackage{footnotehyper}}{\usepackage{footnote}}
\makesavenoteenv{longtable}
\usepackage{graphicx}
\makeatletter
\def\maxwidth{\ifdim\Gin@nat@width>\linewidth\linewidth\else\Gin@nat@width\fi}
\def\maxheight{\ifdim\Gin@nat@height>\textheight\textheight\else\Gin@nat@height\fi}
\makeatother
% Scale images if necessary, so that they will not overflow the page
% margins by default, and it is still possible to overwrite the defaults
% using explicit options in \includegraphics[width, height, ...]{}
\setkeys{Gin}{width=\maxwidth,height=\maxheight,keepaspectratio}
% Set default figure placement to htbp
\makeatletter
\def\fps@figure{htbp}
\makeatother
\setlength{\emergencystretch}{3em} % prevent overfull lines
\providecommand{\tightlist}{%
  \setlength{\itemsep}{0pt}\setlength{\parskip}{0pt}}
\setcounter{secnumdepth}{5}
\usepackage{booktabs}
\usepackage{longtable}
\usepackage{array}
\usepackage{multirow}
\usepackage{wrapfig}
\usepackage{float}
\usepackage{colortbl}
\usepackage{pdflscape}
\usepackage{tabu}
\usepackage{threeparttable}
\usepackage{threeparttablex}
\usepackage[normalem]{ulem}
\usepackage{makecell}
\usepackage{xcolor}
\ifLuaTeX
  \usepackage{selnolig}  % disable illegal ligatures
\fi
\usepackage[]{natbib}
\bibliographystyle{apalike}
\nocite{*}
\IfFileExists{bookmark.sty}{\usepackage{bookmark}}{\usepackage{hyperref}}
\IfFileExists{xurl.sty}{\usepackage{xurl}}{} % add URL line breaks if available
\urlstyle{same} % disable monospaced font for URLs
\hypersetup{
  pdftitle={Supplemental Material for Environmental connectivity influences long-term evolutionary outcomes},
  hidelinks,
  pdfcreator={LaTeX via pandoc}}

\title{Supplemental Material for Environmental connectivity influences long-term evolutionary outcomes}
\author{}
\date{\vspace{-2.5em}2025-08-11}

\begin{document}
\maketitle

{
\setcounter{tocdepth}{1}
\tableofcontents
}
\hypertarget{introduction}{%
\chapter{Introduction}\label{introduction}}

This is the supplemental material our 2025 Artificial Life Conference paper, ``Environmental connectivity influences long-term evolutionary outcomes''.
This is not intended as a stand-alone document, but as a companion to our main manuscript.

\hypertarget{about-our-supplemental-material}{%
\section{About our supplemental material}\label{about-our-supplemental-material}}

Our supplemental material is hosted using \href{https://pages.github.com/}{GitHub pages}.
We compiled our data analyses and supplemental documentation into this web-accessible book using \href{https://bookdown.org}{bookdown}.

The source code and configuration files for this supplemental material can be found in \href{https://github.com/amlalejini/alife-2025-env-conn}{this GitHub repository}.

Our supplemental material includes the following:

\begin{itemize}
\tightlist
\item
  Data availability
  (Section \ref{data-availability})
\item
  Local compilation instructions
  (Section \ref{compilation-instructions})
\item
  TODO
\end{itemize}

\hypertarget{contributing-authors}{%
\section{Contributing authors}\label{contributing-authors}}

\begin{itemize}
\tightlist
\item
  Grant Gordon
\item
  \href{https://fergusonaj.github.io/}{Austin J. Ferguson}
\item
  \href{https://ecodelab.com/}{Emily Dolson}
\item
  \href{https://lalejini.com}{Alexander Lalejini}
\end{itemize}

\hypertarget{data-availability}{%
\chapter{Data availability}\label{data-availability}}

\hypertarget{source-code}{%
\section{Source code}\label{source-code}}

The source code for his work is publicly accessible on GitHub: \url{https://github.com/amlalejini/alife-2025-env-conn}.
This repository has also been archived on Zenodo: \url{https://doi.org/10.5281/zenodo.16795777}

\hypertarget{experiment-results}{%
\section{Experiment results}\label{experiment-results}}

Data generated from our experiments used in analyses are available online, archived in an OSF repository: \url{https://osf.io/ahs6m/}

On OSF, the following compressed archives contain the data presented in our manuscript:

\begin{itemize}
\tightlist
\item
  \texttt{2025-04-17-squished-lattice-longer-avida.tar.gz}
\item
  \texttt{2025-04-17-vary-structs-avida.tar.gz}
\item
  \texttt{squished-lattice-mabe.tar.gz}
\item
  \texttt{vary-structs-mabe.tar.gz}
\end{itemize}

\hypertarget{compilation-instructions}{%
\chapter{Compilation instructions}\label{compilation-instructions}}

Instructions for compiling and running the software used in this study on your local machine.
All experiments were run on Mac or Linux-based operating systems.

You will need a C++ compiler that supports at least C++17.
We used g++13 for all local compilations.

You will also need Python to run graph generation and analysis.
Python dependencies are listed in the \texttt{requirements.txt} at the root of this repository.

Statistical analyses and data visualizations were conducted using R.

Experiments in our simplified model used the MABE2 software, and experiments with digital organisms (self-replicating computer programs) used a modified version of the Avida software platform.

\hypertarget{instructions}{%
\section{Instructions}\label{instructions}}

First, clone the \texttt{alife-2025-env-conn} repository (\url{https://github.com/amlalejini/alife-2025-env-conn.git}) to your machine.
Then, initialize and update git submodule inside the repository.
From inside the repository on your machine, run:

\begin{verbatim}
git submodule update --init --recursive
\end{verbatim}

This will download and update the following dependencies:

\begin{itemize}
\tightlist
\item
  \texttt{avida-empirical} (commit hash: \texttt{266f95f8fcb452655330dab55caa9f1408b49ffa}): A modified implementation of the Avida software that supports the capacity to configure environmental connectivity.
\item
  \texttt{evo\_spatial\_discoveries} (commit hash: \texttt{2c384e93df231125bae83fc6c38d8dc8c64eb6ee}): Contains configurations for MABE2 experiments.
\item
  \texttt{MABE2} (commit hash: \texttt{4f8eb86f997ee89f6d0e0b1144c5be162f4d8d1b}): MABE = ``Modular agent-based evolver'', which is a software platform deigned to empower developers to easily build and customize software for evolutionary computation or artificial life. We used this platform to implement our non-avida experiments.
\item
  \texttt{network\_correlation} (commit hash: \texttt{9d9a07f7436c3569d10eb3b03c6b30e1238c74ef}): Third-party python implementations of various graph statistics and analyses.
\end{itemize}

To compile Avida, navigate into the \texttt{third-party/avida-empirical/} directory and run \texttt{./build\_avida/}.
The compiled executable will be created in the \texttt{third-party/avida-empirical/cbuild/work/} directory.

To compile MABE2, navigate into the \texttt{third-party/MABE2/build} directory and run \texttt{make\ native}.
The compiled executable will be created in the \texttt{third-party/MABE2/build} directory.

Configuration files used Avida experiments can be found in the \texttt{experiments/} directory (within the \texttt{hpc/config} subdirectory for any given experiment).
Configuration files used for MABE2 experiments can be found in \texttt{third-party/evo\_spatial\_discoveries/experiments/}.

\hypertarget{simple-model---varied-spatial-structure-experiment-analyses}{%
\chapter{Simple model - Varied spatial structure experiment analyses}\label{simple-model---varied-spatial-structure-experiment-analyses}}

\hypertarget{dependencies-and-setup}{%
\section{Dependencies and setup}\label{dependencies-and-setup}}

\begin{Shaded}
\begin{Highlighting}[]
\FunctionTok{library}\NormalTok{(tidyverse)}
\FunctionTok{library}\NormalTok{(cowplot)}
\FunctionTok{library}\NormalTok{(RColorBrewer)}
\FunctionTok{library}\NormalTok{(khroma)}
\FunctionTok{library}\NormalTok{(rstatix)}
\FunctionTok{library}\NormalTok{(knitr)}
\FunctionTok{library}\NormalTok{(kableExtra)}
\FunctionTok{library}\NormalTok{(infer)}
\FunctionTok{source}\NormalTok{(}\StringTok{"https://gist.githubusercontent.com/benmarwick/2a1bb0133ff568cbe28d/raw/fb53bd97121f7f9ce947837ef1a4c65a73bffb3f/geom\_flat\_violin.R"}\NormalTok{)}
\end{Highlighting}
\end{Shaded}

\begin{Shaded}
\begin{Highlighting}[]
\CommentTok{\# Check if Rmd is being compiled using bookdown}
\NormalTok{bookdown }\OtherTok{\textless{}{-}} \FunctionTok{exists}\NormalTok{(}\StringTok{"bookdown\_build"}\NormalTok{)}
\end{Highlighting}
\end{Shaded}

\begin{Shaded}
\begin{Highlighting}[]
\NormalTok{experiment\_slug }\OtherTok{\textless{}{-}} \StringTok{"vg{-}experiments"}
\NormalTok{working\_directory }\OtherTok{\textless{}{-}} \FunctionTok{paste}\NormalTok{(}
  \StringTok{"experiments"}\NormalTok{,}
  \StringTok{"mabe2{-}exps"}\NormalTok{,}
\NormalTok{  experiment\_slug,}
  \AttributeTok{sep =} \StringTok{"/"}
\NormalTok{)}
\CommentTok{\# Adjust working directory if being knitted for bookdown build.}
\ControlFlowTok{if}\NormalTok{ (bookdown) \{}
\NormalTok{  working\_directory }\OtherTok{\textless{}{-}} \FunctionTok{paste0}\NormalTok{(}
\NormalTok{    bookdown\_wd\_prefix,}
\NormalTok{    working\_directory}
\NormalTok{  )}
\NormalTok{\}}
\end{Highlighting}
\end{Shaded}

\begin{Shaded}
\begin{Highlighting}[]
\CommentTok{\# Configure our default graphing theme}
\FunctionTok{theme\_set}\NormalTok{(}\FunctionTok{theme\_cowplot}\NormalTok{())}
\CommentTok{\# Create a directory to store plots}
\NormalTok{plot\_dir }\OtherTok{\textless{}{-}} \FunctionTok{paste}\NormalTok{(}
\NormalTok{  working\_directory,}
  \StringTok{"rmd\_plots"}\NormalTok{,}
  \AttributeTok{sep =} \StringTok{"/"}
\NormalTok{)}

\FunctionTok{dir.create}\NormalTok{(}
\NormalTok{  plot\_dir,}
  \AttributeTok{showWarnings =} \ConstantTok{FALSE}
\NormalTok{)}
\end{Highlighting}
\end{Shaded}

\hypertarget{max-organism-data-analyses}{%
\section{Max organism data analyses}\label{max-organism-data-analyses}}

\begin{Shaded}
\begin{Highlighting}[]
\NormalTok{max\_generation }\OtherTok{\textless{}{-}} \DecValTok{100000}
\NormalTok{max\_org\_data\_path }\OtherTok{\textless{}{-}} \FunctionTok{paste}\NormalTok{(}
\NormalTok{  working\_directory,}
  \StringTok{"data"}\NormalTok{,}
  \StringTok{"combined\_max\_org\_data.csv"}\NormalTok{,}
  \AttributeTok{sep =} \StringTok{"/"}
\NormalTok{)}
\CommentTok{\# Data file has time series}
\NormalTok{max\_org\_data\_ts }\OtherTok{\textless{}{-}} \FunctionTok{read\_csv}\NormalTok{(max\_org\_data\_path)}
\NormalTok{max\_org\_data\_ts }\OtherTok{\textless{}{-}}\NormalTok{ max\_org\_data\_ts }\SpecialCharTok{\%\textgreater{}\%}
  \FunctionTok{mutate}\NormalTok{(}
    \AttributeTok{landscape =} \FunctionTok{as.factor}\NormalTok{(landscape),}
    \AttributeTok{structure =} \FunctionTok{as.factor}\NormalTok{(structure),}
\NormalTok{  ) }\SpecialCharTok{\%\textgreater{}\%}
  \FunctionTok{mutate}\NormalTok{(}
    \AttributeTok{valleys\_crossed =} \FunctionTok{case\_when}\NormalTok{(}
\NormalTok{      landscape }\SpecialCharTok{==} \StringTok{"Valley crossing"} \SpecialCharTok{\textasciitilde{}} \FunctionTok{round}\NormalTok{(}\FunctionTok{log}\NormalTok{(fitness, }\AttributeTok{base =} \FloatTok{1.5}\NormalTok{)),}
      \AttributeTok{.default =} \DecValTok{0}
\NormalTok{    )}
\NormalTok{  )}
\CommentTok{\# Get tibble with just final generation}
\NormalTok{max\_org\_data }\OtherTok{\textless{}{-}}\NormalTok{ max\_org\_data\_ts }\SpecialCharTok{\%\textgreater{}\%}
  \FunctionTok{filter}\NormalTok{(generation }\SpecialCharTok{==}\NormalTok{ max\_generation)}
\end{Highlighting}
\end{Shaded}

Check that replicate count for each condition matches expectations.

\begin{Shaded}
\begin{Highlighting}[]
\NormalTok{run\_summary }\OtherTok{\textless{}{-}}\NormalTok{ max\_org\_data }\SpecialCharTok{\%\textgreater{}\%}
  \FunctionTok{group\_by}\NormalTok{(landscape, structure) }\SpecialCharTok{\%\textgreater{}\%}
  \FunctionTok{summarize}\NormalTok{(}
    \AttributeTok{n =} \FunctionTok{n}\NormalTok{()}
\NormalTok{  )}
\FunctionTok{print}\NormalTok{(run\_summary, }\AttributeTok{n =} \DecValTok{30}\NormalTok{)}
\end{Highlighting}
\end{Shaded}

\begin{verbatim}
## # A tibble: 30 x 3
## # Groups:   landscape [3]
##    landscape       structure         n
##    <fct>           <fct>         <int>
##  1 Multipath       clique_ring      50
##  2 Multipath       comet_kite       50
##  3 Multipath       cycle            50
##  4 Multipath       lattice          50
##  5 Multipath       linear_chain     50
##  6 Multipath       random_waxman    50
##  7 Multipath       star             50
##  8 Multipath       well_mixed       50
##  9 Multipath       wheel            50
## 10 Multipath       windmill         50
## 11 Single gradient clique_ring      50
## 12 Single gradient comet_kite       50
## 13 Single gradient cycle            50
## 14 Single gradient lattice          50
## 15 Single gradient linear_chain     50
## 16 Single gradient random_waxman    50
## 17 Single gradient star             50
## 18 Single gradient well_mixed       50
## 19 Single gradient wheel            50
## 20 Single gradient windmill         50
## 21 Valley crossing clique_ring      50
## 22 Valley crossing comet_kite       50
## 23 Valley crossing cycle            50
## 24 Valley crossing lattice          50
## 25 Valley crossing linear_chain     50
## 26 Valley crossing random_waxman    50
## 27 Valley crossing star             50
## 28 Valley crossing well_mixed       50
## 29 Valley crossing wheel            50
## 30 Valley crossing windmill         50
\end{verbatim}

\hypertarget{fitness-in-smooth-gradient-landscape}{%
\subsection{Fitness in smooth gradient landscape}\label{fitness-in-smooth-gradient-landscape}}

Maximum fitness

\begin{Shaded}
\begin{Highlighting}[]
\NormalTok{single\_gradient\_final\_fitness\_plt }\OtherTok{\textless{}{-}} \FunctionTok{ggplot}\NormalTok{(}
    \AttributeTok{data =} \FunctionTok{filter}\NormalTok{(max\_org\_data, landscape }\SpecialCharTok{==} \StringTok{"Single gradient"}\NormalTok{),}
    \AttributeTok{mapping =} \FunctionTok{aes}\NormalTok{(}
      \AttributeTok{x =}\NormalTok{ structure,}
      \AttributeTok{y =}\NormalTok{ fitness,}
      \AttributeTok{fill =}\NormalTok{ structure}
\NormalTok{    )}
\NormalTok{  ) }\SpecialCharTok{+}
  \FunctionTok{geom\_flat\_violin}\NormalTok{(}
    \AttributeTok{position =} \FunctionTok{position\_nudge}\NormalTok{(}\AttributeTok{x =}\NormalTok{ .}\DecValTok{2}\NormalTok{, }\AttributeTok{y =} \DecValTok{0}\NormalTok{),}
    \AttributeTok{alpha =}\NormalTok{ .}\DecValTok{8}
\NormalTok{  ) }\SpecialCharTok{+}
  \FunctionTok{geom\_point}\NormalTok{(}
    \AttributeTok{mapping =} \FunctionTok{aes}\NormalTok{(}\AttributeTok{color =}\NormalTok{ structure),}
    \AttributeTok{position =} \FunctionTok{position\_jitter}\NormalTok{(}\AttributeTok{width =}\NormalTok{ .}\DecValTok{15}\NormalTok{),}
    \AttributeTok{size =}\NormalTok{ .}\DecValTok{5}\NormalTok{,}
    \AttributeTok{alpha =} \FloatTok{0.8}
\NormalTok{  ) }\SpecialCharTok{+}
  \FunctionTok{geom\_boxplot}\NormalTok{(}
    \AttributeTok{width =}\NormalTok{ .}\DecValTok{1}\NormalTok{,}
    \AttributeTok{outlier.shape =} \ConstantTok{NA}\NormalTok{,}
    \AttributeTok{alpha =} \FloatTok{0.5}
\NormalTok{  ) }\SpecialCharTok{+}
  \FunctionTok{theme}\NormalTok{(}
    \AttributeTok{legend.position =} \StringTok{"none"}\NormalTok{,}
    \AttributeTok{axis.text.x =} \FunctionTok{element\_text}\NormalTok{(}
      \AttributeTok{angle =} \DecValTok{30}\NormalTok{,}
      \AttributeTok{hjust =} \DecValTok{1}
\NormalTok{    )}
\NormalTok{  )}

\FunctionTok{ggsave}\NormalTok{(}
  \AttributeTok{filename =} \FunctionTok{paste0}\NormalTok{(plot\_dir, }\StringTok{"/single\_gradient\_final\_fitness.pdf"}\NormalTok{),}
  \AttributeTok{plot =}\NormalTok{ single\_gradient\_final\_fitness\_plt,}
  \AttributeTok{width =} \DecValTok{15}\NormalTok{,}
  \AttributeTok{height =} \DecValTok{10}
\NormalTok{)}

\NormalTok{single\_gradient\_final\_fitness\_plt}
\end{Highlighting}
\end{Shaded}

\includegraphics{supplemental-material_files/figure-latex/unnamed-chunk-8-1.pdf}

Maximum fitness over time

\begin{Shaded}
\begin{Highlighting}[]
\NormalTok{single\_gradient\_fitness\_ts\_plt }\OtherTok{\textless{}{-}} \FunctionTok{ggplot}\NormalTok{(}
    \AttributeTok{data =} \FunctionTok{filter}\NormalTok{(max\_org\_data\_ts, landscape }\SpecialCharTok{==} \StringTok{"Single gradient"}\NormalTok{),}
    \AttributeTok{mapping =} \FunctionTok{aes}\NormalTok{(}
      \AttributeTok{x =}\NormalTok{ generation,}
      \AttributeTok{y =}\NormalTok{ fitness,}
      \AttributeTok{color =}\NormalTok{ structure,}
      \AttributeTok{fill =}\NormalTok{ structure}
\NormalTok{    )}
\NormalTok{  ) }\SpecialCharTok{+}
  \FunctionTok{stat\_summary}\NormalTok{(}\AttributeTok{fun =} \StringTok{"mean"}\NormalTok{, }\AttributeTok{geom =} \StringTok{"line"}\NormalTok{) }\SpecialCharTok{+}
  \FunctionTok{stat\_summary}\NormalTok{(}
    \AttributeTok{fun.data =} \StringTok{"mean\_cl\_boot"}\NormalTok{,}
    \AttributeTok{fun.args =} \FunctionTok{list}\NormalTok{(}\AttributeTok{conf.int =} \FloatTok{0.95}\NormalTok{),}
    \AttributeTok{geom =} \StringTok{"ribbon"}\NormalTok{,}
    \AttributeTok{alpha =} \FloatTok{0.2}\NormalTok{,}
    \AttributeTok{linetype =} \DecValTok{0}
\NormalTok{  ) }\SpecialCharTok{+}
  \FunctionTok{theme}\NormalTok{(}\AttributeTok{legend.position =} \StringTok{"bottom"}\NormalTok{)}

\FunctionTok{ggsave}\NormalTok{(}
  \AttributeTok{plot =}\NormalTok{ single\_gradient\_fitness\_ts\_plt,}
  \AttributeTok{filename =} \FunctionTok{paste0}\NormalTok{(}
\NormalTok{    plot\_dir,}
    \StringTok{"/single\_gradient\_fitness\_ts.pdf"}
\NormalTok{  ),}
  \AttributeTok{width =} \DecValTok{15}\NormalTok{,}
  \AttributeTok{height =} \DecValTok{10}
\NormalTok{)}

\NormalTok{single\_gradient\_fitness\_ts\_plt}
\end{Highlighting}
\end{Shaded}

\includegraphics{supplemental-material_files/figure-latex/unnamed-chunk-9-1.pdf}

Time to maximum fitness

\begin{Shaded}
\begin{Highlighting}[]
\CommentTok{\# Find all rows with maximum fitness value, then get row with minimum generation,}
\CommentTok{\#  then project out expected generation to max (for runs that didn\textquotesingle{}t finish)}
\NormalTok{max\_possible\_fit }\OtherTok{=} \DecValTok{50}
\NormalTok{time\_to\_max\_single\_gradient }\OtherTok{\textless{}{-}}\NormalTok{ max\_org\_data\_ts }\SpecialCharTok{\%\textgreater{}\%}
  \FunctionTok{filter}\NormalTok{(landscape }\SpecialCharTok{==} \StringTok{"Single gradient"}\NormalTok{) }\SpecialCharTok{\%\textgreater{}\%}
  \FunctionTok{group\_by}\NormalTok{(rep, structure) }\SpecialCharTok{\%\textgreater{}\%}
  \FunctionTok{slice\_max}\NormalTok{(}
\NormalTok{    fitness,}
    \AttributeTok{n =} \DecValTok{1}
\NormalTok{  ) }\SpecialCharTok{\%\textgreater{}\%}
  \FunctionTok{slice\_min}\NormalTok{(}
\NormalTok{    generation,}
    \AttributeTok{n =} \DecValTok{1}
\NormalTok{  ) }\SpecialCharTok{\%\textgreater{}\%}
  \FunctionTok{mutate}\NormalTok{(}
    \AttributeTok{proj\_gen\_max =}\NormalTok{ (max\_possible\_fit }\SpecialCharTok{/}\NormalTok{ fitness) }\SpecialCharTok{*}\NormalTok{ generation}
\NormalTok{  )}
\end{Highlighting}
\end{Shaded}

\begin{Shaded}
\begin{Highlighting}[]
\NormalTok{single\_gradient\_gen\_max\_proj\_plt }\OtherTok{\textless{}{-}} \FunctionTok{ggplot}\NormalTok{(}
    \AttributeTok{data =}\NormalTok{ time\_to\_max\_single\_gradient,}
    \AttributeTok{mapping =} \FunctionTok{aes}\NormalTok{(}
      \AttributeTok{x =}\NormalTok{ structure,}
      \AttributeTok{y =}\NormalTok{ proj\_gen\_max,}
      \AttributeTok{fill =}\NormalTok{ structure}
\NormalTok{    )}
\NormalTok{  ) }\SpecialCharTok{+}
  \FunctionTok{geom\_flat\_violin}\NormalTok{(}
    \AttributeTok{position =} \FunctionTok{position\_nudge}\NormalTok{(}\AttributeTok{x =}\NormalTok{ .}\DecValTok{2}\NormalTok{, }\AttributeTok{y =} \DecValTok{0}\NormalTok{),}
    \AttributeTok{alpha =}\NormalTok{ .}\DecValTok{8}
\NormalTok{  ) }\SpecialCharTok{+}
  \FunctionTok{geom\_point}\NormalTok{(}
    \AttributeTok{mapping =} \FunctionTok{aes}\NormalTok{(}\AttributeTok{color =}\NormalTok{ structure),}
    \AttributeTok{position =} \FunctionTok{position\_jitter}\NormalTok{(}\AttributeTok{width =}\NormalTok{ .}\DecValTok{15}\NormalTok{),}
    \AttributeTok{size =}\NormalTok{ .}\DecValTok{5}\NormalTok{,}
    \AttributeTok{alpha =} \FloatTok{0.8}
\NormalTok{  ) }\SpecialCharTok{+}
  \FunctionTok{geom\_boxplot}\NormalTok{(}
    \AttributeTok{width =}\NormalTok{ .}\DecValTok{1}\NormalTok{,}
    \AttributeTok{outlier.shape =} \ConstantTok{NA}\NormalTok{,}
    \AttributeTok{alpha =} \FloatTok{0.5}
\NormalTok{  ) }\SpecialCharTok{+}
  \FunctionTok{scale\_y\_log10}\NormalTok{(}
    \AttributeTok{guide =} \StringTok{"axis\_logticks"}
\NormalTok{  ) }\SpecialCharTok{+}
  \CommentTok{\# scale\_y\_continuous(}
  \CommentTok{\#   trans="pseudo\_log",}
  \CommentTok{\#   breaks = c(10, 100, 1000, 10000, 100000, 1000000)}
  \CommentTok{\#   ,limits = c(10, 100, 1000, 10000, 100000, 1000000)}
  \CommentTok{\# ) +}
  \FunctionTok{geom\_hline}\NormalTok{(}
    \AttributeTok{yintercept =}\NormalTok{ max\_generation,}
    \AttributeTok{linetype =} \StringTok{"dashed"}
\NormalTok{  ) }\SpecialCharTok{+}
  \FunctionTok{theme}\NormalTok{(}
    \AttributeTok{legend.position =} \StringTok{"none"}\NormalTok{,}
    \AttributeTok{axis.text.x =} \FunctionTok{element\_text}\NormalTok{(}
      \AttributeTok{angle =} \DecValTok{30}\NormalTok{,}
      \AttributeTok{hjust =} \DecValTok{1}
\NormalTok{    )}
\NormalTok{  )}

\FunctionTok{ggsave}\NormalTok{(}
  \AttributeTok{filename =} \FunctionTok{paste0}\NormalTok{(plot\_dir, }\StringTok{"/single\_gradient\_gen\_max\_proj.pdf"}\NormalTok{),}
  \AttributeTok{plot =}\NormalTok{ single\_gradient\_gen\_max\_proj\_plt,}
  \AttributeTok{width =} \DecValTok{15}\NormalTok{,}
  \AttributeTok{height =} \DecValTok{10}
\NormalTok{)}

\NormalTok{single\_gradient\_gen\_max\_proj\_plt}
\end{Highlighting}
\end{Shaded}

\includegraphics{supplemental-material_files/figure-latex/unnamed-chunk-11-1.pdf}

Rank ordering of time to max fitness values

\begin{Shaded}
\begin{Highlighting}[]
\NormalTok{time\_to\_max\_single\_gradient }\SpecialCharTok{\%\textgreater{}\%}
  \FunctionTok{group\_by}\NormalTok{(structure) }\SpecialCharTok{\%\textgreater{}\%}
  \FunctionTok{summarize}\NormalTok{(}
    \AttributeTok{reps =} \FunctionTok{n}\NormalTok{(),}
    \AttributeTok{median\_proj\_gen =} \FunctionTok{median}\NormalTok{(proj\_gen\_max),}
    \AttributeTok{mean\_proj\_gen =} \FunctionTok{mean}\NormalTok{(proj\_gen\_max)}
\NormalTok{  ) }\SpecialCharTok{\%\textgreater{}\%}
  \FunctionTok{arrange}\NormalTok{(}
\NormalTok{    mean\_proj\_gen}
\NormalTok{  )}
\end{Highlighting}
\end{Shaded}

\begin{verbatim}
## # A tibble: 10 x 4
##    structure      reps median_proj_gen mean_proj_gen
##    <fct>         <int>           <dbl>         <dbl>
##  1 well_mixed       50          18000         18240 
##  2 random_waxman    50          18000         18260 
##  3 comet_kite       50          21000         21220 
##  4 windmill         50          26000         26100 
##  5 lattice          50          27000         27460 
##  6 clique_ring      50          36000         36020 
##  7 cycle            50          69000         68840 
##  8 linear_chain     50          69000         69080 
##  9 wheel            50         135481.       135502.
## 10 star             50         361785.       366603.
\end{verbatim}

\begin{Shaded}
\begin{Highlighting}[]
\FunctionTok{kruskal.test}\NormalTok{(}
  \AttributeTok{formula =}\NormalTok{ proj\_gen\_max }\SpecialCharTok{\textasciitilde{}}\NormalTok{ structure,}
  \AttributeTok{data =}\NormalTok{ time\_to\_max\_single\_gradient}
\NormalTok{)}
\end{Highlighting}
\end{Shaded}

\begin{verbatim}
## 
##  Kruskal-Wallis rank sum test
## 
## data:  proj_gen_max by structure
## Kruskal-Wallis chi-squared = 490.93, df = 9, p-value < 2.2e-16
\end{verbatim}

\begin{Shaded}
\begin{Highlighting}[]
\NormalTok{wc\_results }\OtherTok{\textless{}{-}} \FunctionTok{pairwise.wilcox.test}\NormalTok{(}
  \AttributeTok{x =}\NormalTok{ time\_to\_max\_single\_gradient}\SpecialCharTok{$}\NormalTok{proj\_gen\_max,}
  \AttributeTok{g =}\NormalTok{ time\_to\_max\_single\_gradient}\SpecialCharTok{$}\NormalTok{structure,}
  \AttributeTok{p.adjust.method   =} \StringTok{"holm"}\NormalTok{,}
  \AttributeTok{exact =} \ConstantTok{FALSE}
\NormalTok{)}

\NormalTok{single\_gradient\_proj\_gen\_max\_wc\_table }\OtherTok{\textless{}{-}} \FunctionTok{kbl}\NormalTok{(wc\_results}\SpecialCharTok{$}\NormalTok{p.value) }\SpecialCharTok{\%\textgreater{}\%}
  \FunctionTok{kable\_styling}\NormalTok{()}

\FunctionTok{save\_kable}\NormalTok{(}
\NormalTok{  single\_gradient\_proj\_gen\_max\_wc\_table,}
  \FunctionTok{paste0}\NormalTok{(plot\_dir, }\StringTok{"/single\_gradient\_proj\_gen\_max\_wc\_table.pdf"}\NormalTok{)}
\NormalTok{)}

\NormalTok{single\_gradient\_proj\_gen\_max\_wc\_table}
\end{Highlighting}
\end{Shaded}

\begin{table}
\centering
\begin{tabular}[t]{l|r|r|r|r|r|r|r|r|r}
\hline
  & clique\_ring & comet\_kite & cycle & lattice & linear\_chain & random\_waxman & star & well\_mixed & wheel\\
\hline
comet\_kite & 0 & NA & NA & NA & NA & NA & NA & NA & NA\\
\hline
cycle & 0 & 0 & NA & NA & NA & NA & NA & NA & NA\\
\hline
lattice & 0 & 0 & 0.0000000 & NA & NA & NA & NA & NA & NA\\
\hline
linear\_chain & 0 & 0 & 0.2915242 & 0 & NA & NA & NA & NA & NA\\
\hline
random\_waxman & 0 & 0 & 0.0000000 & 0 & 0 & NA & NA & NA & NA\\
\hline
star & 0 & 0 & 0.0000000 & 0 & 0 & 0.0000000 & NA & NA & NA\\
\hline
well\_mixed & 0 & 0 & 0.0000000 & 0 & 0 & 0.8218339 & 0 & NA & NA\\
\hline
wheel & 0 & 0 & 0.0000000 & 0 & 0 & 0.0000000 & 0 & 0 & NA\\
\hline
windmill & 0 & 0 & 0.0000000 & 0 & 0 & 0.0000000 & 0 & 0 & 0\\
\hline
\end{tabular}
\end{table}

\begin{Shaded}
\begin{Highlighting}[]
\FunctionTok{library}\NormalTok{(boot)}
\CommentTok{\# Define sample mean function}
\NormalTok{samplemean }\OtherTok{\textless{}{-}} \ControlFlowTok{function}\NormalTok{(x, d) \{}
  \FunctionTok{return}\NormalTok{(}\FunctionTok{mean}\NormalTok{(x[d]))}
\NormalTok{\}}

\NormalTok{summary\_gen\_to\_max }\OtherTok{\textless{}{-}} \FunctionTok{tibble}\NormalTok{(}
  \AttributeTok{structure =} \FunctionTok{character}\NormalTok{(),}
  \AttributeTok{proj\_gen\_max\_mean =} \FunctionTok{double}\NormalTok{(),}
  \AttributeTok{proj\_gen\_max\_mean\_ci\_low =} \FunctionTok{double}\NormalTok{(),}
  \AttributeTok{proj\_gen\_max\_mean\_ci\_high =} \FunctionTok{double}\NormalTok{()}
\NormalTok{)}

\NormalTok{structures }\OtherTok{\textless{}{-}} \FunctionTok{levels}\NormalTok{(time\_to\_max\_single\_gradient}\SpecialCharTok{$}\NormalTok{structure)}
\ControlFlowTok{for}\NormalTok{ (struct }\ControlFlowTok{in}\NormalTok{ structures) \{}
\NormalTok{  boot\_result }\OtherTok{\textless{}{-}} \FunctionTok{boot}\NormalTok{(}
    \AttributeTok{data =} \FunctionTok{filter}\NormalTok{(}
\NormalTok{      time\_to\_max\_single\_gradient,}
\NormalTok{      structure }\SpecialCharTok{==}\NormalTok{ struct}
\NormalTok{    )}\SpecialCharTok{$}\NormalTok{proj\_gen\_max,}
    \AttributeTok{statistic =}\NormalTok{ samplemean,}
    \AttributeTok{R =} \DecValTok{10000}
\NormalTok{  )}
\NormalTok{  result\_ci }\OtherTok{\textless{}{-}} \FunctionTok{boot.ci}\NormalTok{(boot\_result, }\AttributeTok{conf =} \FloatTok{0.99}\NormalTok{, }\AttributeTok{type =} \StringTok{"perc"}\NormalTok{)}
\NormalTok{  m }\OtherTok{\textless{}{-}}\NormalTok{ result\_ci}\SpecialCharTok{$}\NormalTok{t0}
\NormalTok{  low }\OtherTok{\textless{}{-}}\NormalTok{ result\_ci}\SpecialCharTok{$}\NormalTok{percent[}\DecValTok{4}\NormalTok{]}
\NormalTok{  high }\OtherTok{\textless{}{-}}\NormalTok{ result\_ci}\SpecialCharTok{$}\NormalTok{percent[}\DecValTok{5}\NormalTok{]}

\NormalTok{  summary\_gen\_to\_max }\OtherTok{\textless{}{-}}\NormalTok{ summary\_gen\_to\_max }\SpecialCharTok{\%\textgreater{}\%}
    \FunctionTok{add\_row}\NormalTok{(}
      \AttributeTok{structure =}\NormalTok{ struct,}
      \AttributeTok{proj\_gen\_max\_mean =}\NormalTok{ m,}
      \AttributeTok{proj\_gen\_max\_mean\_ci\_low =}\NormalTok{ low,}
      \AttributeTok{proj\_gen\_max\_mean\_ci\_high =}\NormalTok{ high}
\NormalTok{    )}
\NormalTok{\}}

\NormalTok{wm\_median }\OtherTok{\textless{}{-}} \FunctionTok{median}\NormalTok{(}
  \FunctionTok{filter}\NormalTok{(time\_to\_max\_single\_gradient, structure }\SpecialCharTok{==} \StringTok{"well\_mixed"}\NormalTok{)}\SpecialCharTok{$}\NormalTok{proj\_gen\_max}
\NormalTok{)}

\NormalTok{simple\_time\_to\_max\_plt }\OtherTok{\textless{}{-}} \FunctionTok{ggplot}\NormalTok{(}
    \AttributeTok{data =}\NormalTok{ summary\_gen\_to\_max,}
    \AttributeTok{mapping =} \FunctionTok{aes}\NormalTok{(}
      \AttributeTok{x =}\NormalTok{ structure,}
      \AttributeTok{y =}\NormalTok{ proj\_gen\_max\_mean,}
      \AttributeTok{fill =}\NormalTok{ structure,}
      \AttributeTok{color =}\NormalTok{ structure}
\NormalTok{    )}
\NormalTok{  ) }\SpecialCharTok{+}
  \CommentTok{\# geom\_point() +}
  \FunctionTok{geom\_col}\NormalTok{() }\SpecialCharTok{+}
  \FunctionTok{geom\_linerange}\NormalTok{(}
    \FunctionTok{aes}\NormalTok{(}
      \AttributeTok{ymin =}\NormalTok{ proj\_gen\_max\_mean\_ci\_low,}
      \AttributeTok{ymax =}\NormalTok{ proj\_gen\_max\_mean\_ci\_high}
\NormalTok{    ),}
    \AttributeTok{color =} \StringTok{"black"}\NormalTok{,}
    \AttributeTok{linewidth =} \FloatTok{0.75}\NormalTok{,}
    \AttributeTok{lineend =} \StringTok{"round"}
\NormalTok{  ) }\SpecialCharTok{+}
  \CommentTok{\# scale\_y\_log10(}
  \CommentTok{\#   guide = "axis\_logticks"}
  \CommentTok{\# ) +}
  \FunctionTok{geom\_hline}\NormalTok{(}
    \AttributeTok{yintercept =}\NormalTok{ max\_generation,}
    \AttributeTok{linetype =} \StringTok{"dashed"}
\NormalTok{  ) }\SpecialCharTok{+}
  \FunctionTok{geom\_hline}\NormalTok{(}
    \AttributeTok{yintercept =}\NormalTok{ wm\_median,}
    \AttributeTok{linetype =} \StringTok{"dotted"}\NormalTok{,}
    \AttributeTok{color =} \StringTok{"orange"}
\NormalTok{  ) }\SpecialCharTok{+}
  \FunctionTok{scale\_color\_discreterainbow}\NormalTok{() }\SpecialCharTok{+}
  \FunctionTok{scale\_fill\_discreterainbow}\NormalTok{() }\SpecialCharTok{+}
  \FunctionTok{coord\_flip}\NormalTok{() }\SpecialCharTok{+}
  \FunctionTok{theme}\NormalTok{(}
    \AttributeTok{legend.position =} \StringTok{"none"}\NormalTok{,}
    \AttributeTok{axis.text.x =} \FunctionTok{element\_text}\NormalTok{(}
      \AttributeTok{angle =} \DecValTok{30}\NormalTok{,}
      \AttributeTok{hjust =} \DecValTok{1}
\NormalTok{    )}
\NormalTok{  )}

\FunctionTok{ggsave}\NormalTok{(}
  \AttributeTok{filename =} \FunctionTok{paste0}\NormalTok{(plot\_dir, }\StringTok{"/simple\_time\_to\_max.pdf"}\NormalTok{),}
  \AttributeTok{plot =}\NormalTok{ simple\_time\_to\_max\_plt,}
  \AttributeTok{width =} \DecValTok{8}\NormalTok{,}
  \AttributeTok{height =} \DecValTok{4}
\NormalTok{)}

\NormalTok{simple\_time\_to\_max\_plt}
\end{Highlighting}
\end{Shaded}

\includegraphics{supplemental-material_files/figure-latex/unnamed-chunk-14-1.pdf}

\hypertarget{fitness-in-multi-path-landscape}{%
\subsection{Fitness in multi-path landscape}\label{fitness-in-multi-path-landscape}}

\begin{Shaded}
\begin{Highlighting}[]
\NormalTok{multipath\_final\_fitness\_plt }\OtherTok{\textless{}{-}} \FunctionTok{ggplot}\NormalTok{(}
    \AttributeTok{data =} \FunctionTok{filter}\NormalTok{(max\_org\_data, landscape }\SpecialCharTok{==} \StringTok{"Multipath"}\NormalTok{),}
    \AttributeTok{mapping =} \FunctionTok{aes}\NormalTok{(}
      \AttributeTok{x =}\NormalTok{ structure,}
      \AttributeTok{y =}\NormalTok{ fitness,}
      \AttributeTok{fill =}\NormalTok{ structure}
\NormalTok{    )}
\NormalTok{  ) }\SpecialCharTok{+}
  \CommentTok{\# geom\_flat\_violin(}
  \CommentTok{\#   position = position\_nudge(x = .2, y = 0),}
  \CommentTok{\#   alpha = .8}
  \CommentTok{\# ) +}
  \FunctionTok{geom\_point}\NormalTok{(}
    \AttributeTok{mapping =} \FunctionTok{aes}\NormalTok{(}\AttributeTok{color =}\NormalTok{ structure),}
    \AttributeTok{position =} \FunctionTok{position\_jitter}\NormalTok{(}\AttributeTok{width =}\NormalTok{ .}\DecValTok{15}\NormalTok{),}
    \AttributeTok{size =}\NormalTok{ .}\DecValTok{5}\NormalTok{,}
    \AttributeTok{alpha =} \FloatTok{0.8}
\NormalTok{  ) }\SpecialCharTok{+}
  \FunctionTok{geom\_boxplot}\NormalTok{(}
    \AttributeTok{width =}\NormalTok{ .}\DecValTok{3}\NormalTok{,}
    \AttributeTok{outlier.shape =} \ConstantTok{NA}\NormalTok{,}
    \AttributeTok{alpha =} \FloatTok{0.5}
\NormalTok{  ) }\SpecialCharTok{+}
  \FunctionTok{scale\_color\_discreterainbow}\NormalTok{() }\SpecialCharTok{+}
  \FunctionTok{scale\_fill\_discreterainbow}\NormalTok{() }\SpecialCharTok{+}
  \FunctionTok{theme}\NormalTok{(}
    \AttributeTok{legend.position =} \StringTok{"none"}\NormalTok{,}
    \AttributeTok{axis.text.x =} \FunctionTok{element\_text}\NormalTok{(}
      \AttributeTok{angle =} \DecValTok{30}\NormalTok{,}
      \AttributeTok{hjust =} \DecValTok{1}
\NormalTok{    )}
\NormalTok{  )}

\FunctionTok{ggsave}\NormalTok{(}
  \AttributeTok{filename =} \FunctionTok{paste0}\NormalTok{(plot\_dir, }\StringTok{"/multipath\_final\_fitness.pdf"}\NormalTok{),}
  \AttributeTok{plot =}\NormalTok{ multipath\_final\_fitness\_plt,}
  \AttributeTok{width =} \DecValTok{6}\NormalTok{,}
  \AttributeTok{height =} \DecValTok{4}
\NormalTok{)}

\NormalTok{multipath\_final\_fitness\_plt}
\end{Highlighting}
\end{Shaded}

\includegraphics{supplemental-material_files/figure-latex/unnamed-chunk-15-1.pdf}

Max fitness over time

\begin{Shaded}
\begin{Highlighting}[]
\NormalTok{multipath\_fitness\_ts\_plt }\OtherTok{\textless{}{-}} \FunctionTok{ggplot}\NormalTok{(}
    \AttributeTok{data =} \FunctionTok{filter}\NormalTok{(max\_org\_data\_ts, landscape }\SpecialCharTok{==} \StringTok{"Multipath"}\NormalTok{),}
    \AttributeTok{mapping =} \FunctionTok{aes}\NormalTok{(}
      \AttributeTok{x =}\NormalTok{ generation,}
      \AttributeTok{y =}\NormalTok{ fitness,}
      \AttributeTok{color =}\NormalTok{ structure,}
      \AttributeTok{fill =}\NormalTok{ structure}
\NormalTok{    )}
\NormalTok{  ) }\SpecialCharTok{+}
  \FunctionTok{stat\_summary}\NormalTok{(}\AttributeTok{fun =} \StringTok{"mean"}\NormalTok{, }\AttributeTok{geom =} \StringTok{"line"}\NormalTok{) }\SpecialCharTok{+}
  \FunctionTok{stat\_summary}\NormalTok{(}
    \AttributeTok{fun.data =} \StringTok{"mean\_cl\_boot"}\NormalTok{,}
    \AttributeTok{fun.args =} \FunctionTok{list}\NormalTok{(}\AttributeTok{conf.int =} \FloatTok{0.95}\NormalTok{),}
    \AttributeTok{geom =} \StringTok{"ribbon"}\NormalTok{,}
    \AttributeTok{alpha =} \FloatTok{0.2}\NormalTok{,}
    \AttributeTok{linetype =} \DecValTok{0}
\NormalTok{  ) }\SpecialCharTok{+}
  \FunctionTok{theme}\NormalTok{(}\AttributeTok{legend.position =} \StringTok{"bottom"}\NormalTok{)}

\FunctionTok{ggsave}\NormalTok{(}
  \AttributeTok{plot =}\NormalTok{ multipath\_fitness\_ts\_plt,}
  \AttributeTok{filename =} \FunctionTok{paste0}\NormalTok{(}
\NormalTok{    plot\_dir,}
    \StringTok{"/multipath\_fitness\_ts.pdf"}
\NormalTok{  ),}
  \AttributeTok{width =} \DecValTok{15}\NormalTok{,}
  \AttributeTok{height =} \DecValTok{10}
\NormalTok{)}

\NormalTok{multipath\_fitness\_ts\_plt}
\end{Highlighting}
\end{Shaded}

\includegraphics{supplemental-material_files/figure-latex/unnamed-chunk-16-1.pdf}

Rank ordering of fitness values

\begin{Shaded}
\begin{Highlighting}[]
\NormalTok{max\_org\_data }\SpecialCharTok{\%\textgreater{}\%}
  \FunctionTok{filter}\NormalTok{(landscape }\SpecialCharTok{==} \StringTok{"Multipath"}\NormalTok{) }\SpecialCharTok{\%\textgreater{}\%}
  \FunctionTok{group\_by}\NormalTok{(structure) }\SpecialCharTok{\%\textgreater{}\%}
  \FunctionTok{summarize}\NormalTok{(}
    \AttributeTok{reps =} \FunctionTok{n}\NormalTok{(),}
    \AttributeTok{median\_fitness =} \FunctionTok{median}\NormalTok{(fitness),}
    \AttributeTok{mean\_fitness =} \FunctionTok{mean}\NormalTok{(fitness)}
\NormalTok{  ) }\SpecialCharTok{\%\textgreater{}\%}
  \FunctionTok{arrange}\NormalTok{(}
    \FunctionTok{desc}\NormalTok{(mean\_fitness)}
\NormalTok{  )}
\end{Highlighting}
\end{Shaded}

\begin{verbatim}
## # A tibble: 10 x 4
##    structure      reps median_fitness mean_fitness
##    <fct>         <int>          <dbl>        <dbl>
##  1 linear_chain     50           4.86         4.80
##  2 cycle            50           4.88         4.79
##  3 clique_ring      50           4.84         4.79
##  4 lattice          50           3.46         3.38
##  5 windmill         50           3.4          3.34
##  6 wheel            50           3.28         3.27
##  7 well_mixed       50           3.34         3.25
##  8 random_waxman    50           3.14         3.23
##  9 star             50           3.08         3.17
## 10 comet_kite       50           2.94         3.06
\end{verbatim}

\begin{Shaded}
\begin{Highlighting}[]
\FunctionTok{kruskal.test}\NormalTok{(}
  \AttributeTok{formula =}\NormalTok{ fitness }\SpecialCharTok{\textasciitilde{}}\NormalTok{ structure,}
  \AttributeTok{data =} \FunctionTok{filter}\NormalTok{(max\_org\_data, landscape }\SpecialCharTok{==} \StringTok{"Multipath"}\NormalTok{)}
\NormalTok{)}
\end{Highlighting}
\end{Shaded}

\begin{verbatim}
## 
##  Kruskal-Wallis rank sum test
## 
## data:  fitness by structure
## Kruskal-Wallis chi-squared = 246.11, df = 9, p-value < 2.2e-16
\end{verbatim}

\begin{Shaded}
\begin{Highlighting}[]
\NormalTok{wc\_results }\OtherTok{\textless{}{-}} \FunctionTok{pairwise.wilcox.test}\NormalTok{(}
  \AttributeTok{x =} \FunctionTok{filter}\NormalTok{(max\_org\_data, landscape }\SpecialCharTok{==} \StringTok{"Multipath"}\NormalTok{)}\SpecialCharTok{$}\NormalTok{fitness,}
  \AttributeTok{g =} \FunctionTok{filter}\NormalTok{(max\_org\_data, landscape }\SpecialCharTok{==} \StringTok{"Multipath"}\NormalTok{)}\SpecialCharTok{$}\NormalTok{structure,}
  \AttributeTok{p.adjust.method   =} \StringTok{"holm"}\NormalTok{,}
  \AttributeTok{exact =} \ConstantTok{FALSE}
\NormalTok{)}

\NormalTok{mp\_fitness\_wc\_table }\OtherTok{\textless{}{-}} \FunctionTok{kbl}\NormalTok{(wc\_results}\SpecialCharTok{$}\NormalTok{p.value) }\SpecialCharTok{\%\textgreater{}\%}
  \FunctionTok{kable\_styling}\NormalTok{()}

\FunctionTok{save\_kable}\NormalTok{(}
\NormalTok{  mp\_fitness\_wc\_table,}
  \FunctionTok{paste0}\NormalTok{(plot\_dir, }\StringTok{"/multipath\_fitness\_wc\_table.pdf"}\NormalTok{)}
\NormalTok{)}

\NormalTok{mp\_fitness\_wc\_table}
\end{Highlighting}
\end{Shaded}

\begin{table}
\centering
\begin{tabular}[t]{l|r|r|r|r|r|r|r|r|r}
\hline
  & clique\_ring & comet\_kite & cycle & lattice & linear\_chain & random\_waxman & star & well\_mixed & wheel\\
\hline
comet\_kite & 0 & NA & NA & NA & NA & NA & NA & NA & NA\\
\hline
cycle & 1 & 0 & NA & NA & NA & NA & NA & NA & NA\\
\hline
lattice & 0 & 1 & 0 & NA & NA & NA & NA & NA & NA\\
\hline
linear\_chain & 1 & 0 & 1 & 0 & NA & NA & NA & NA & NA\\
\hline
random\_waxman & 0 & 1 & 0 & 1 & 0 & NA & NA & NA & NA\\
\hline
star & 0 & 1 & 0 & 1 & 0 & 1 & NA & NA & NA\\
\hline
well\_mixed & 0 & 1 & 0 & 1 & 0 & 1 & 1 & NA & NA\\
\hline
wheel & 0 & 1 & 0 & 1 & 0 & 1 & 1 & 1 & NA\\
\hline
windmill & 0 & 1 & 0 & 1 & 0 & 1 & 1 & 1 & 1\\
\hline
\end{tabular}
\end{table}

\hypertarget{valleys-crossed-in-valley-crossing-landscape}{%
\subsection{Valleys crossed in valley-crossing landscape}\label{valleys-crossed-in-valley-crossing-landscape}}

\begin{Shaded}
\begin{Highlighting}[]
\NormalTok{valleycrossing\_valleys\_plt }\OtherTok{\textless{}{-}} \FunctionTok{ggplot}\NormalTok{(}
    \AttributeTok{data =} \FunctionTok{filter}\NormalTok{(max\_org\_data, landscape }\SpecialCharTok{==} \StringTok{"Valley crossing"}\NormalTok{),}
    \AttributeTok{mapping =} \FunctionTok{aes}\NormalTok{(}
      \AttributeTok{x =}\NormalTok{ structure,}
      \AttributeTok{y =}\NormalTok{ valleys\_crossed,}
      \AttributeTok{fill =}\NormalTok{ structure}
\NormalTok{    )}
\NormalTok{  ) }\SpecialCharTok{+}
  \CommentTok{\# geom\_flat\_violin(}
  \CommentTok{\#   position = position\_nudge(x = .2, y = 0),}
  \CommentTok{\#   alpha = .8}
  \CommentTok{\# ) +}
  \FunctionTok{geom\_point}\NormalTok{(}
    \AttributeTok{mapping =} \FunctionTok{aes}\NormalTok{(}\AttributeTok{color =}\NormalTok{ structure),}
    \AttributeTok{position =} \FunctionTok{position\_jitter}\NormalTok{(}\AttributeTok{width =}\NormalTok{ .}\DecValTok{15}\NormalTok{),}
    \AttributeTok{size =}\NormalTok{ .}\DecValTok{5}\NormalTok{,}
    \AttributeTok{alpha =} \FloatTok{0.8}
\NormalTok{  ) }\SpecialCharTok{+}
  \FunctionTok{geom\_boxplot}\NormalTok{(}
    \AttributeTok{width =}\NormalTok{ .}\DecValTok{3}\NormalTok{,}
    \AttributeTok{outlier.shape =} \ConstantTok{NA}\NormalTok{,}
    \AttributeTok{alpha =} \FloatTok{0.5}
\NormalTok{  ) }\SpecialCharTok{+}
  \FunctionTok{scale\_color\_discreterainbow}\NormalTok{() }\SpecialCharTok{+}
  \FunctionTok{scale\_fill\_discreterainbow}\NormalTok{() }\SpecialCharTok{+}
  \FunctionTok{theme}\NormalTok{(}
    \AttributeTok{legend.position =} \StringTok{"none"}\NormalTok{,}
    \AttributeTok{axis.text.x =} \FunctionTok{element\_text}\NormalTok{(}
      \AttributeTok{angle =} \DecValTok{30}\NormalTok{,}
      \AttributeTok{hjust =} \DecValTok{1}
\NormalTok{    )}
\NormalTok{  )}

\FunctionTok{ggsave}\NormalTok{(}
  \AttributeTok{filename =} \FunctionTok{paste0}\NormalTok{(plot\_dir, }\StringTok{"/valleycrossing\_valleys\_crossed.pdf"}\NormalTok{),}
  \AttributeTok{plot =}\NormalTok{ valleycrossing\_valleys\_plt,}
  \AttributeTok{width =} \DecValTok{6}\NormalTok{,}
  \AttributeTok{height =} \DecValTok{4}
\NormalTok{)}

\NormalTok{valleycrossing\_valleys\_plt}
\end{Highlighting}
\end{Shaded}

\includegraphics{supplemental-material_files/figure-latex/unnamed-chunk-19-1.pdf}

Rank ordering of fitness values

\begin{Shaded}
\begin{Highlighting}[]
\NormalTok{vc }\OtherTok{\textless{}{-}}\NormalTok{ max\_org\_data }\SpecialCharTok{\%\textgreater{}\%}
  \FunctionTok{filter}\NormalTok{(landscape }\SpecialCharTok{==} \StringTok{"Valley crossing"}\NormalTok{) }\SpecialCharTok{\%\textgreater{}\%}
  \FunctionTok{group\_by}\NormalTok{(structure) }\SpecialCharTok{\%\textgreater{}\%}
  \FunctionTok{summarize}\NormalTok{(}
    \AttributeTok{reps =} \FunctionTok{n}\NormalTok{(),}
    \AttributeTok{median\_valleys\_crossed =} \FunctionTok{median}\NormalTok{(valleys\_crossed),}
    \AttributeTok{mean\_valleys\_crossed =} \FunctionTok{mean}\NormalTok{(valleys\_crossed),}
    \AttributeTok{min\_valleys\_crossed =} \FunctionTok{min}\NormalTok{(valleys\_crossed)}
\NormalTok{  ) }\SpecialCharTok{\%\textgreater{}\%}
  \FunctionTok{arrange}\NormalTok{(}
    \FunctionTok{desc}\NormalTok{(mean\_valleys\_crossed)}
\NormalTok{  )}
\NormalTok{vc}
\end{Highlighting}
\end{Shaded}

\begin{verbatim}
## # A tibble: 10 x 5
##    structure      reps median_valleys_crossed mean_valleys_crossed
##    <fct>         <int>                  <dbl>                <dbl>
##  1 cycle            50                    100               100   
##  2 linear_chain     50                    100               100   
##  3 lattice          50                     41                41.9 
##  4 star             50                     21                21.5 
##  5 comet_kite       50                     20                20.5 
##  6 windmill         50                     11                11.6 
##  7 clique_ring      50                     10                10.3 
##  8 random_waxman    50                      9                 8.76
##  9 well_mixed       50                      8                 8.46
## 10 wheel            50                      6                 6.6 
## # i 1 more variable: min_valleys_crossed <dbl>
\end{verbatim}

\begin{Shaded}
\begin{Highlighting}[]
\NormalTok{vc}\SpecialCharTok{$}\NormalTok{min\_valleys\_crossed}
\end{Highlighting}
\end{Shaded}

\begin{verbatim}
##  [1] 100 100  28  12  13   5   5   3   4   1
\end{verbatim}

\begin{Shaded}
\begin{Highlighting}[]
\FunctionTok{kruskal.test}\NormalTok{(}
  \AttributeTok{formula =}\NormalTok{ valleys\_crossed }\SpecialCharTok{\textasciitilde{}}\NormalTok{ structure,}
  \AttributeTok{data =} \FunctionTok{filter}\NormalTok{(max\_org\_data, landscape }\SpecialCharTok{==} \StringTok{"Valley crossing"}\NormalTok{)}
\NormalTok{)}
\end{Highlighting}
\end{Shaded}

\begin{verbatim}
## 
##  Kruskal-Wallis rank sum test
## 
## data:  valleys_crossed by structure
## Kruskal-Wallis chi-squared = 444.04, df = 9, p-value < 2.2e-16
\end{verbatim}

\begin{Shaded}
\begin{Highlighting}[]
\NormalTok{wc\_results }\OtherTok{\textless{}{-}} \FunctionTok{pairwise.wilcox.test}\NormalTok{(}
  \AttributeTok{x =} \FunctionTok{filter}\NormalTok{(max\_org\_data, landscape }\SpecialCharTok{==} \StringTok{"Valley crossing"}\NormalTok{)}\SpecialCharTok{$}\NormalTok{valleys\_crossed,}
  \AttributeTok{g =} \FunctionTok{filter}\NormalTok{(max\_org\_data, landscape }\SpecialCharTok{==} \StringTok{"Valley crossing"}\NormalTok{)}\SpecialCharTok{$}\NormalTok{structure,}
  \AttributeTok{p.adjust.method   =} \StringTok{"holm"}\NormalTok{,}
  \AttributeTok{exact =} \ConstantTok{FALSE}
\NormalTok{)}

\NormalTok{vc\_valleys\_crossed\_wc\_table }\OtherTok{\textless{}{-}} \FunctionTok{kbl}\NormalTok{(wc\_results}\SpecialCharTok{$}\NormalTok{p.value) }\SpecialCharTok{\%\textgreater{}\%}
  \FunctionTok{kable\_styling}\NormalTok{()}

\FunctionTok{save\_kable}\NormalTok{(}
\NormalTok{  vc\_valleys\_crossed\_wc\_table,}
  \FunctionTok{paste0}\NormalTok{(plot\_dir, }\StringTok{"/valley\_crossing\_valleys\_wc\_table.pdf"}\NormalTok{)}
\NormalTok{)}

\NormalTok{vc\_valleys\_crossed\_wc\_table}
\end{Highlighting}
\end{Shaded}

\begin{table}
\centering
\begin{tabular}[t]{l|r|r|r|r|r|r|r|r|r}
\hline
  & clique\_ring & comet\_kite & cycle & lattice & linear\_chain & random\_waxman & star & well\_mixed & wheel\\
\hline
comet\_kite & 0.0000000 & NA & NA & NA & NA & NA & NA & NA & NA\\
\hline
cycle & 0.0000000 & 0.0000000 & NA & NA & NA & NA & NA & NA & NA\\
\hline
lattice & 0.0000000 & 0.0000000 & 0 & NA & NA & NA & NA & NA & NA\\
\hline
linear\_chain & 0.0000000 & 0.0000000 & NaN & 0 & NA & NA & NA & NA & NA\\
\hline
random\_waxman & 0.0414336 & 0.0000000 & 0 & 0 & 0 & NA & NA & NA & NA\\
\hline
star & 0.0000000 & 0.4016992 & 0 & 0 & 0 & 0.0000000 & NA & NA & NA\\
\hline
well\_mixed & 0.0028498 & 0.0000000 & 0 & 0 & 0 & 0.4620430 & 0 & NA & NA\\
\hline
wheel & 0.0000001 & 0.0000000 & 0 & 0 & 0 & 0.0029961 & 0 & 0.0119690 & NA\\
\hline
windmill & 0.0895493 & 0.0000000 & 0 & 0 & 0 & 0.0001323 & 0 & 0.0000028 & 0\\
\hline
\end{tabular}
\end{table}

\hypertarget{simple-model---squished-toroid-experiment-analyses}{%
\chapter{Simple model - Squished toroid experiment analyses}\label{simple-model---squished-toroid-experiment-analyses}}

\hypertarget{setup-and-dependencies}{%
\section{Setup and Dependencies}\label{setup-and-dependencies}}

\begin{Shaded}
\begin{Highlighting}[]
\FunctionTok{library}\NormalTok{(tidyverse)}
\FunctionTok{library}\NormalTok{(cowplot)}
\FunctionTok{library}\NormalTok{(RColorBrewer)}
\FunctionTok{library}\NormalTok{(khroma)}
\FunctionTok{library}\NormalTok{(rstatix)}
\FunctionTok{library}\NormalTok{(knitr)}
\FunctionTok{library}\NormalTok{(kableExtra)}
\FunctionTok{library}\NormalTok{(infer)}
\FunctionTok{source}\NormalTok{(}\StringTok{"https://gist.githubusercontent.com/benmarwick/2a1bb0133ff568cbe28d/raw/fb53bd97121f7f9ce947837ef1a4c65a73bffb3f/geom\_flat\_violin.R"}\NormalTok{)}
\end{Highlighting}
\end{Shaded}

\begin{Shaded}
\begin{Highlighting}[]
\CommentTok{\# Check if Rmd is being compiled using bookdown}
\NormalTok{bookdown }\OtherTok{\textless{}{-}} \FunctionTok{exists}\NormalTok{(}\StringTok{"bookdown\_build"}\NormalTok{)}
\end{Highlighting}
\end{Shaded}

\begin{Shaded}
\begin{Highlighting}[]
\NormalTok{experiment\_slug }\OtherTok{\textless{}{-}} \StringTok{"lattice{-}experiments"}
\NormalTok{working\_directory }\OtherTok{\textless{}{-}} \FunctionTok{paste}\NormalTok{(}
  \StringTok{"experiments"}\NormalTok{,}
  \StringTok{"mabe2{-}exps"}\NormalTok{,}
\NormalTok{  experiment\_slug,}
  \AttributeTok{sep =} \StringTok{"/"}
\NormalTok{)}
\CommentTok{\# Adjust working directory if being knitted for bookdown build.}
\ControlFlowTok{if}\NormalTok{ (bookdown) \{}
\NormalTok{  working\_directory }\OtherTok{\textless{}{-}} \FunctionTok{paste0}\NormalTok{(}
\NormalTok{    bookdown\_wd\_prefix,}
\NormalTok{    working\_directory}
\NormalTok{  )}
\NormalTok{\}}
\end{Highlighting}
\end{Shaded}

\begin{Shaded}
\begin{Highlighting}[]
\CommentTok{\# Configure our default graphing theme}
\FunctionTok{theme\_set}\NormalTok{(}\FunctionTok{theme\_cowplot}\NormalTok{())}
\CommentTok{\# Create a directory to store plots}
\NormalTok{plot\_dir }\OtherTok{\textless{}{-}} \FunctionTok{paste}\NormalTok{(}
\NormalTok{  working\_directory,}
  \StringTok{"rmd\_plots"}\NormalTok{,}
  \AttributeTok{sep =} \StringTok{"/"}
\NormalTok{)}

\FunctionTok{dir.create}\NormalTok{(}
\NormalTok{  plot\_dir,}
  \AttributeTok{showWarnings =} \ConstantTok{FALSE}
\NormalTok{)}
\end{Highlighting}
\end{Shaded}

\hypertarget{max-organism-data-analyses-1}{%
\section{Max organism data analyses}\label{max-organism-data-analyses-1}}

\begin{Shaded}
\begin{Highlighting}[]
\NormalTok{max\_generation }\OtherTok{\textless{}{-}} \DecValTok{100000}
\NormalTok{max\_org\_data\_path }\OtherTok{\textless{}{-}} \FunctionTok{paste}\NormalTok{(}
\NormalTok{  working\_directory,}
  \StringTok{"data"}\NormalTok{,}
  \StringTok{"combined\_max\_org\_data.csv"}\NormalTok{,}
  \AttributeTok{sep =} \StringTok{"/"}
\NormalTok{)}
\CommentTok{\# Data file has time series}
\NormalTok{max\_org\_data\_ts }\OtherTok{\textless{}{-}} \FunctionTok{read\_csv}\NormalTok{(max\_org\_data\_path)}
\NormalTok{max\_org\_data\_ts }\OtherTok{\textless{}{-}}\NormalTok{ max\_org\_data\_ts }\SpecialCharTok{\%\textgreater{}\%}
  \FunctionTok{mutate}\NormalTok{(}
    \AttributeTok{landscape =} \FunctionTok{as.factor}\NormalTok{(landscape),}
    \AttributeTok{structure =} \FunctionTok{factor}\NormalTok{(}
\NormalTok{      structure,}
      \AttributeTok{levels =} \FunctionTok{c}\NormalTok{(}
        \StringTok{"1\_3600"}\NormalTok{,}
        \StringTok{"2\_1800"}\NormalTok{,}
        \StringTok{"3\_1200"}\NormalTok{,}
        \StringTok{"4\_900"}\NormalTok{,}
        \StringTok{"15\_240"}\NormalTok{,}
        \StringTok{"30\_120"}\NormalTok{,}
        \StringTok{"60\_60"}
\NormalTok{      )}
\NormalTok{    ),}
\NormalTok{  ) }\SpecialCharTok{\%\textgreater{}\%}
  \FunctionTok{mutate}\NormalTok{(}
    \AttributeTok{valleys\_crossed =} \FunctionTok{case\_when}\NormalTok{(}
\NormalTok{      landscape }\SpecialCharTok{==} \StringTok{"Valley crossing"} \SpecialCharTok{\textasciitilde{}} \FunctionTok{round}\NormalTok{(}\FunctionTok{log}\NormalTok{(fitness, }\AttributeTok{base =} \FloatTok{1.5}\NormalTok{)),}
      \AttributeTok{.default =} \DecValTok{0}
\NormalTok{    )}
\NormalTok{  )}
\CommentTok{\# Get tibble with just final generation}
\NormalTok{max\_org\_data }\OtherTok{\textless{}{-}}\NormalTok{ max\_org\_data\_ts }\SpecialCharTok{\%\textgreater{}\%}
  \FunctionTok{filter}\NormalTok{(generation }\SpecialCharTok{==}\NormalTok{ max\_generation)}
\end{Highlighting}
\end{Shaded}

Check that replicate count for each condition matches expectations.

\begin{Shaded}
\begin{Highlighting}[]
\NormalTok{run\_summary }\OtherTok{\textless{}{-}}\NormalTok{ max\_org\_data }\SpecialCharTok{\%\textgreater{}\%}
  \FunctionTok{group\_by}\NormalTok{(landscape, structure) }\SpecialCharTok{\%\textgreater{}\%}
  \FunctionTok{summarize}\NormalTok{(}
    \AttributeTok{n =} \FunctionTok{n}\NormalTok{()}
\NormalTok{  )}
\FunctionTok{print}\NormalTok{(run\_summary, }\AttributeTok{n =} \DecValTok{30}\NormalTok{)}
\end{Highlighting}
\end{Shaded}

\begin{verbatim}
## # A tibble: 21 x 3
## # Groups:   landscape [3]
##    landscape       structure     n
##    <fct>           <fct>     <int>
##  1 Multipath       1_3600       50
##  2 Multipath       2_1800       50
##  3 Multipath       3_1200       50
##  4 Multipath       4_900        50
##  5 Multipath       15_240       50
##  6 Multipath       30_120       50
##  7 Multipath       60_60        50
##  8 Single gradient 1_3600       50
##  9 Single gradient 2_1800       50
## 10 Single gradient 3_1200       50
## 11 Single gradient 4_900        50
## 12 Single gradient 15_240       50
## 13 Single gradient 30_120       50
## 14 Single gradient 60_60        50
## 15 Valley crossing 1_3600       50
## 16 Valley crossing 2_1800       50
## 17 Valley crossing 3_1200       50
## 18 Valley crossing 4_900        50
## 19 Valley crossing 15_240       50
## 20 Valley crossing 30_120       50
## 21 Valley crossing 60_60        50
\end{verbatim}

\hypertarget{fitness-in-smooth-gradient-landscape-1}{%
\subsection{Fitness in smooth gradient landscape}\label{fitness-in-smooth-gradient-landscape-1}}

\begin{Shaded}
\begin{Highlighting}[]
\NormalTok{single\_gradient\_final\_fitness\_plt }\OtherTok{\textless{}{-}} \FunctionTok{ggplot}\NormalTok{(}
    \AttributeTok{data =} \FunctionTok{filter}\NormalTok{(max\_org\_data, landscape }\SpecialCharTok{==} \StringTok{"Single gradient"}\NormalTok{),}
    \AttributeTok{mapping =} \FunctionTok{aes}\NormalTok{(}
      \AttributeTok{x =}\NormalTok{ structure,}
      \AttributeTok{y =}\NormalTok{ fitness,}
      \AttributeTok{fill =}\NormalTok{ structure}
\NormalTok{    )}
\NormalTok{  ) }\SpecialCharTok{+}
  \FunctionTok{geom\_flat\_violin}\NormalTok{(}
    \AttributeTok{position =} \FunctionTok{position\_nudge}\NormalTok{(}\AttributeTok{x =}\NormalTok{ .}\DecValTok{2}\NormalTok{, }\AttributeTok{y =} \DecValTok{0}\NormalTok{),}
    \AttributeTok{alpha =}\NormalTok{ .}\DecValTok{8}
\NormalTok{  ) }\SpecialCharTok{+}
  \FunctionTok{geom\_point}\NormalTok{(}
    \AttributeTok{mapping =} \FunctionTok{aes}\NormalTok{(}\AttributeTok{color =}\NormalTok{ structure),}
    \AttributeTok{position =} \FunctionTok{position\_jitter}\NormalTok{(}\AttributeTok{width =}\NormalTok{ .}\DecValTok{15}\NormalTok{),}
    \AttributeTok{size =}\NormalTok{ .}\DecValTok{5}\NormalTok{,}
    \AttributeTok{alpha =} \FloatTok{0.8}
\NormalTok{  ) }\SpecialCharTok{+}
  \FunctionTok{geom\_boxplot}\NormalTok{(}
    \AttributeTok{width =}\NormalTok{ .}\DecValTok{1}\NormalTok{,}
    \AttributeTok{outlier.shape =} \ConstantTok{NA}\NormalTok{,}
    \AttributeTok{alpha =} \FloatTok{0.5}
\NormalTok{  ) }\SpecialCharTok{+}
  \FunctionTok{theme}\NormalTok{(}
    \AttributeTok{legend.position =} \StringTok{"none"}\NormalTok{,}
    \AttributeTok{axis.text.x =} \FunctionTok{element\_text}\NormalTok{(}
      \AttributeTok{angle =} \DecValTok{30}\NormalTok{,}
      \AttributeTok{hjust =} \DecValTok{1}
\NormalTok{    )}
\NormalTok{  )}

\FunctionTok{ggsave}\NormalTok{(}
  \AttributeTok{filename =} \FunctionTok{paste0}\NormalTok{(plot\_dir, }\StringTok{"/single\_gradient\_final\_fitness.pdf"}\NormalTok{),}
  \AttributeTok{plot =}\NormalTok{ single\_gradient\_final\_fitness\_plt,}
  \AttributeTok{width =} \DecValTok{15}\NormalTok{,}
  \AttributeTok{height =} \DecValTok{10}
\NormalTok{)}

\NormalTok{single\_gradient\_final\_fitness\_plt}
\end{Highlighting}
\end{Shaded}

\includegraphics{supplemental-material_files/figure-latex/unnamed-chunk-28-1.pdf}

Max fitness over time

\begin{Shaded}
\begin{Highlighting}[]
\NormalTok{single\_gradient\_fitness\_ts\_plt }\OtherTok{\textless{}{-}} \FunctionTok{ggplot}\NormalTok{(}
    \AttributeTok{data =} \FunctionTok{filter}\NormalTok{(max\_org\_data\_ts, landscape }\SpecialCharTok{==} \StringTok{"Single gradient"}\NormalTok{),}
    \AttributeTok{mapping =} \FunctionTok{aes}\NormalTok{(}
      \AttributeTok{x =}\NormalTok{ generation,}
      \AttributeTok{y =}\NormalTok{ fitness,}
      \AttributeTok{color =}\NormalTok{ structure,}
      \AttributeTok{fill =}\NormalTok{ structure}
\NormalTok{    )}
\NormalTok{  ) }\SpecialCharTok{+}
  \FunctionTok{stat\_summary}\NormalTok{(}\AttributeTok{fun =} \StringTok{"mean"}\NormalTok{, }\AttributeTok{geom =} \StringTok{"line"}\NormalTok{) }\SpecialCharTok{+}
  \FunctionTok{stat\_summary}\NormalTok{(}
    \AttributeTok{fun.data =} \StringTok{"mean\_cl\_boot"}\NormalTok{,}
    \AttributeTok{fun.args =} \FunctionTok{list}\NormalTok{(}\AttributeTok{conf.int =} \FloatTok{0.95}\NormalTok{),}
    \AttributeTok{geom =} \StringTok{"ribbon"}\NormalTok{,}
    \AttributeTok{alpha =} \FloatTok{0.2}\NormalTok{,}
    \AttributeTok{linetype =} \DecValTok{0}
\NormalTok{  ) }\SpecialCharTok{+}
  \FunctionTok{theme}\NormalTok{(}\AttributeTok{legend.position =} \StringTok{"bottom"}\NormalTok{)}

\FunctionTok{ggsave}\NormalTok{(}
  \AttributeTok{plot =}\NormalTok{ single\_gradient\_fitness\_ts\_plt,}
  \AttributeTok{filename =} \FunctionTok{paste0}\NormalTok{(}
\NormalTok{    plot\_dir,}
    \StringTok{"/single\_gradient\_fitness\_ts.pdf"}
\NormalTok{  ),}
  \AttributeTok{width =} \DecValTok{15}\NormalTok{,}
  \AttributeTok{height =} \DecValTok{10}
\NormalTok{)}

\NormalTok{single\_gradient\_fitness\_ts\_plt}
\end{Highlighting}
\end{Shaded}

\includegraphics{supplemental-material_files/figure-latex/unnamed-chunk-29-1.pdf}

Time to maximum fitness

\begin{Shaded}
\begin{Highlighting}[]
\CommentTok{\# Find all rows with maximum fitness value, then get row with minimum generation,}
\CommentTok{\#  then project out expected generation to max (for runs that didn\textquotesingle{}t finish)}
\NormalTok{max\_possible\_fit }\OtherTok{=} \DecValTok{50}
\NormalTok{time\_to\_max\_single\_gradient }\OtherTok{\textless{}{-}}\NormalTok{ max\_org\_data\_ts }\SpecialCharTok{\%\textgreater{}\%}
  \FunctionTok{filter}\NormalTok{(landscape }\SpecialCharTok{==} \StringTok{"Single gradient"}\NormalTok{) }\SpecialCharTok{\%\textgreater{}\%}
  \FunctionTok{group\_by}\NormalTok{(rep, structure) }\SpecialCharTok{\%\textgreater{}\%}
  \FunctionTok{slice\_max}\NormalTok{(}
\NormalTok{    fitness,}
    \AttributeTok{n =} \DecValTok{1}
\NormalTok{  ) }\SpecialCharTok{\%\textgreater{}\%}
  \FunctionTok{slice\_min}\NormalTok{(}
\NormalTok{    generation,}
    \AttributeTok{n =} \DecValTok{1}
\NormalTok{  ) }\SpecialCharTok{\%\textgreater{}\%}
  \FunctionTok{mutate}\NormalTok{(}
    \AttributeTok{proj\_gen\_max =}\NormalTok{ (max\_possible\_fit }\SpecialCharTok{/}\NormalTok{ fitness) }\SpecialCharTok{*}\NormalTok{ generation}
\NormalTok{  )}
\end{Highlighting}
\end{Shaded}

\begin{Shaded}
\begin{Highlighting}[]
\NormalTok{single\_gradient\_gen\_max\_proj\_plt }\OtherTok{\textless{}{-}} \FunctionTok{ggplot}\NormalTok{(}
    \AttributeTok{data =}\NormalTok{ time\_to\_max\_single\_gradient,}
    \AttributeTok{mapping =} \FunctionTok{aes}\NormalTok{(}
      \AttributeTok{x =}\NormalTok{ structure,}
      \AttributeTok{y =}\NormalTok{ proj\_gen\_max,}
      \AttributeTok{fill =}\NormalTok{ structure}
\NormalTok{    )}
\NormalTok{  ) }\SpecialCharTok{+}
  \FunctionTok{geom\_flat\_violin}\NormalTok{(}
    \AttributeTok{position =} \FunctionTok{position\_nudge}\NormalTok{(}\AttributeTok{x =}\NormalTok{ .}\DecValTok{2}\NormalTok{, }\AttributeTok{y =} \DecValTok{0}\NormalTok{),}
    \AttributeTok{alpha =}\NormalTok{ .}\DecValTok{8}
\NormalTok{  ) }\SpecialCharTok{+}
  \FunctionTok{geom\_point}\NormalTok{(}
    \AttributeTok{mapping =} \FunctionTok{aes}\NormalTok{(}\AttributeTok{color =}\NormalTok{ structure),}
    \AttributeTok{position =} \FunctionTok{position\_jitter}\NormalTok{(}\AttributeTok{width =}\NormalTok{ .}\DecValTok{15}\NormalTok{),}
    \AttributeTok{size =}\NormalTok{ .}\DecValTok{5}\NormalTok{,}
    \AttributeTok{alpha =} \FloatTok{0.8}
\NormalTok{  ) }\SpecialCharTok{+}
  \FunctionTok{geom\_boxplot}\NormalTok{(}
    \AttributeTok{width =}\NormalTok{ .}\DecValTok{1}\NormalTok{,}
    \AttributeTok{outlier.shape =} \ConstantTok{NA}\NormalTok{,}
    \AttributeTok{alpha =} \FloatTok{0.5}
\NormalTok{  ) }\SpecialCharTok{+}
  \FunctionTok{scale\_y\_log10}\NormalTok{(}
    \AttributeTok{guide =} \StringTok{"axis\_logticks"}
\NormalTok{  ) }\SpecialCharTok{+}
  \CommentTok{\# scale\_y\_continuous(}
  \CommentTok{\#   trans="pseudo\_log",}
  \CommentTok{\#   breaks = c(10, 100, 1000, 10000, 100000, 1000000)}
  \CommentTok{\#   ,limits = c(10, 100, 1000, 10000, 100000, 1000000)}
  \CommentTok{\# ) +}
  \FunctionTok{geom\_hline}\NormalTok{(}
    \AttributeTok{yintercept =}\NormalTok{ max\_generation,}
    \AttributeTok{linetype =} \StringTok{"dashed"}
\NormalTok{  ) }\SpecialCharTok{+}
  \FunctionTok{theme}\NormalTok{(}
    \AttributeTok{legend.position =} \StringTok{"none"}\NormalTok{,}
    \AttributeTok{axis.text.x =} \FunctionTok{element\_text}\NormalTok{(}
      \AttributeTok{angle =} \DecValTok{30}\NormalTok{,}
      \AttributeTok{hjust =} \DecValTok{1}
\NormalTok{    )}
\NormalTok{  )}

\FunctionTok{ggsave}\NormalTok{(}
  \AttributeTok{filename =} \FunctionTok{paste0}\NormalTok{(plot\_dir, }\StringTok{"/single\_gradient\_gen\_max\_proj.pdf"}\NormalTok{),}
  \AttributeTok{plot =}\NormalTok{ single\_gradient\_gen\_max\_proj\_plt,}
  \AttributeTok{width =} \DecValTok{15}\NormalTok{,}
  \AttributeTok{height =} \DecValTok{10}
\NormalTok{)}

\NormalTok{single\_gradient\_gen\_max\_proj\_plt}
\end{Highlighting}
\end{Shaded}

\includegraphics{supplemental-material_files/figure-latex/unnamed-chunk-31-1.pdf}

Rank ordering of time to max fitness values

\begin{Shaded}
\begin{Highlighting}[]
\NormalTok{time\_to\_max\_single\_gradient }\SpecialCharTok{\%\textgreater{}\%}
  \FunctionTok{group\_by}\NormalTok{(structure) }\SpecialCharTok{\%\textgreater{}\%}
  \FunctionTok{summarize}\NormalTok{(}
    \AttributeTok{reps =} \FunctionTok{n}\NormalTok{(),}
    \AttributeTok{median\_proj\_gen =} \FunctionTok{median}\NormalTok{(proj\_gen\_max),}
    \AttributeTok{mean\_proj\_gen =} \FunctionTok{mean}\NormalTok{(proj\_gen\_max)}
\NormalTok{  ) }\SpecialCharTok{\%\textgreater{}\%}
  \FunctionTok{arrange}\NormalTok{(}
\NormalTok{    mean\_proj\_gen}
\NormalTok{  )}
\end{Highlighting}
\end{Shaded}

\begin{verbatim}
## # A tibble: 7 x 4
##   structure  reps median_proj_gen mean_proj_gen
##   <fct>     <int>           <dbl>         <dbl>
## 1 60_60        50           28000         27540
## 2 30_120       50           28000         27880
## 3 15_240       50           30000         30160
## 4 4_900        50           42000         42020
## 5 3_1200       50           47000         46900
## 6 2_1800       50           53000         53340
## 7 1_3600       50           69000         68700
\end{verbatim}

\begin{Shaded}
\begin{Highlighting}[]
\FunctionTok{kruskal.test}\NormalTok{(}
  \AttributeTok{formula =}\NormalTok{ proj\_gen\_max }\SpecialCharTok{\textasciitilde{}}\NormalTok{ structure,}
  \AttributeTok{data =}\NormalTok{ time\_to\_max\_single\_gradient}
\NormalTok{)}
\end{Highlighting}
\end{Shaded}

\begin{verbatim}
## 
##  Kruskal-Wallis rank sum test
## 
## data:  proj_gen_max by structure
## Kruskal-Wallis chi-squared = 341.17, df = 6, p-value < 2.2e-16
\end{verbatim}

\begin{Shaded}
\begin{Highlighting}[]
\NormalTok{wc\_results }\OtherTok{\textless{}{-}} \FunctionTok{pairwise.wilcox.test}\NormalTok{(}
  \AttributeTok{x =}\NormalTok{ time\_to\_max\_single\_gradient}\SpecialCharTok{$}\NormalTok{proj\_gen\_max,}
  \AttributeTok{g =}\NormalTok{ time\_to\_max\_single\_gradient}\SpecialCharTok{$}\NormalTok{structure,}
  \AttributeTok{p.adjust.method   =} \StringTok{"holm"}\NormalTok{,}
  \AttributeTok{exact =} \ConstantTok{FALSE}
\NormalTok{)}

\NormalTok{single\_gradient\_proj\_gen\_max\_wc\_table }\OtherTok{\textless{}{-}} \FunctionTok{kbl}\NormalTok{(wc\_results}\SpecialCharTok{$}\NormalTok{p.value) }\SpecialCharTok{\%\textgreater{}\%}
  \FunctionTok{kable\_styling}\NormalTok{()}

\FunctionTok{save\_kable}\NormalTok{(}
\NormalTok{  single\_gradient\_proj\_gen\_max\_wc\_table,}
  \FunctionTok{paste0}\NormalTok{(plot\_dir, }\StringTok{"/single\_gradient\_proj\_gen\_max\_wc\_table.pdf"}\NormalTok{)}
\NormalTok{)}
\NormalTok{single\_gradient\_proj\_gen\_max\_wc\_table}
\end{Highlighting}
\end{Shaded}

\begin{table}
\centering
\begin{tabular}[t]{l|r|r|r|r|r|r}
\hline
  & 1\_3600 & 2\_1800 & 3\_1200 & 4\_900 & 15\_240 & 30\_120\\
\hline
2\_1800 & 0 & NA & NA & NA & NA & NA\\
\hline
3\_1200 & 0 & 0 & NA & NA & NA & NA\\
\hline
4\_900 & 0 & 0 & 0 & NA & NA & NA\\
\hline
15\_240 & 0 & 0 & 0 & 0 & NA & NA\\
\hline
30\_120 & 0 & 0 & 0 & 0 & 0 & NA\\
\hline
60\_60 & 0 & 0 & 0 & 0 & 0 & 0.0001966\\
\hline
\end{tabular}
\end{table}

\begin{Shaded}
\begin{Highlighting}[]
\FunctionTok{library}\NormalTok{(boot)}
\CommentTok{\# Define sample mean function}
\NormalTok{samplemean }\OtherTok{\textless{}{-}} \ControlFlowTok{function}\NormalTok{(x, d) \{}
  \FunctionTok{return}\NormalTok{(}\FunctionTok{mean}\NormalTok{(x[d]))}
\NormalTok{\}}

\NormalTok{summary\_gen\_to\_max }\OtherTok{\textless{}{-}} \FunctionTok{tibble}\NormalTok{(}
  \AttributeTok{structure =} \FunctionTok{character}\NormalTok{(),}
  \AttributeTok{proj\_gen\_max\_mean =} \FunctionTok{double}\NormalTok{(),}
  \AttributeTok{proj\_gen\_max\_mean\_ci\_low =} \FunctionTok{double}\NormalTok{(),}
  \AttributeTok{proj\_gen\_max\_mean\_ci\_high =} \FunctionTok{double}\NormalTok{()}
\NormalTok{)}

\NormalTok{structures }\OtherTok{\textless{}{-}} \FunctionTok{levels}\NormalTok{(time\_to\_max\_single\_gradient}\SpecialCharTok{$}\NormalTok{structure)}
\ControlFlowTok{for}\NormalTok{ (struct }\ControlFlowTok{in}\NormalTok{ structures) \{}
\NormalTok{  boot\_result }\OtherTok{\textless{}{-}} \FunctionTok{boot}\NormalTok{(}
    \AttributeTok{data =} \FunctionTok{filter}\NormalTok{(}
\NormalTok{      time\_to\_max\_single\_gradient,}
\NormalTok{      structure }\SpecialCharTok{==}\NormalTok{ struct}
\NormalTok{    )}\SpecialCharTok{$}\NormalTok{proj\_gen\_max,}
    \AttributeTok{statistic =}\NormalTok{ samplemean,}
    \AttributeTok{R =} \DecValTok{10000}
\NormalTok{  )}
\NormalTok{  result\_ci }\OtherTok{\textless{}{-}} \FunctionTok{boot.ci}\NormalTok{(boot\_result, }\AttributeTok{conf =} \FloatTok{0.99}\NormalTok{, }\AttributeTok{type =} \StringTok{"perc"}\NormalTok{)}
\NormalTok{  m }\OtherTok{\textless{}{-}}\NormalTok{ result\_ci}\SpecialCharTok{$}\NormalTok{t0}
\NormalTok{  low }\OtherTok{\textless{}{-}}\NormalTok{ result\_ci}\SpecialCharTok{$}\NormalTok{percent[}\DecValTok{4}\NormalTok{]}
\NormalTok{  high }\OtherTok{\textless{}{-}}\NormalTok{ result\_ci}\SpecialCharTok{$}\NormalTok{percent[}\DecValTok{5}\NormalTok{]}

\NormalTok{  summary\_gen\_to\_max }\OtherTok{\textless{}{-}}\NormalTok{ summary\_gen\_to\_max }\SpecialCharTok{\%\textgreater{}\%}
    \FunctionTok{add\_row}\NormalTok{(}
      \AttributeTok{structure =}\NormalTok{ struct,}
      \AttributeTok{proj\_gen\_max\_mean =}\NormalTok{ m,}
      \AttributeTok{proj\_gen\_max\_mean\_ci\_low =}\NormalTok{ low,}
      \AttributeTok{proj\_gen\_max\_mean\_ci\_high =}\NormalTok{ high}
\NormalTok{    )}
\NormalTok{\}}

\NormalTok{wm\_median }\OtherTok{\textless{}{-}} \FunctionTok{median}\NormalTok{(}
  \FunctionTok{filter}\NormalTok{(time\_to\_max\_single\_gradient, structure }\SpecialCharTok{==} \StringTok{"well\_mixed"}\NormalTok{)}\SpecialCharTok{$}\NormalTok{proj\_gen\_max}
\NormalTok{)}

\NormalTok{simple\_time\_to\_max\_plt }\OtherTok{\textless{}{-}} \FunctionTok{ggplot}\NormalTok{(}
    \AttributeTok{data =}\NormalTok{ summary\_gen\_to\_max,}
    \AttributeTok{mapping =} \FunctionTok{aes}\NormalTok{(}
      \AttributeTok{x =}\NormalTok{ structure,}
      \AttributeTok{y =}\NormalTok{ proj\_gen\_max\_mean,}
      \AttributeTok{fill =}\NormalTok{ structure,}
      \AttributeTok{color =}\NormalTok{ structure}
\NormalTok{    )}
\NormalTok{  ) }\SpecialCharTok{+}
  \CommentTok{\# geom\_point() +}
  \FunctionTok{geom\_col}\NormalTok{() }\SpecialCharTok{+}
  \FunctionTok{geom\_linerange}\NormalTok{(}
    \FunctionTok{aes}\NormalTok{(}
      \AttributeTok{ymin =}\NormalTok{ proj\_gen\_max\_mean\_ci\_low,}
      \AttributeTok{ymax =}\NormalTok{ proj\_gen\_max\_mean\_ci\_high}
\NormalTok{    ),}
    \AttributeTok{color =} \StringTok{"black"}\NormalTok{,}
    \AttributeTok{linewidth =} \FloatTok{0.75}\NormalTok{,}
    \AttributeTok{lineend =} \StringTok{"round"}
\NormalTok{  ) }\SpecialCharTok{+}
  \CommentTok{\# scale\_y\_log10(}
  \CommentTok{\#   guide = "axis\_logticks"}
  \CommentTok{\# ) +}
  \FunctionTok{geom\_hline}\NormalTok{(}
    \AttributeTok{yintercept =}\NormalTok{ max\_generation,}
    \AttributeTok{linetype =} \StringTok{"dashed"}
\NormalTok{  ) }\SpecialCharTok{+}
  \FunctionTok{geom\_hline}\NormalTok{(}
    \AttributeTok{yintercept =}\NormalTok{ wm\_median,}
    \AttributeTok{linetype =} \StringTok{"dotted"}\NormalTok{,}
    \AttributeTok{color =} \StringTok{"orange"}
\NormalTok{  ) }\SpecialCharTok{+}
  \FunctionTok{scale\_color\_discreterainbow}\NormalTok{() }\SpecialCharTok{+}
  \FunctionTok{scale\_fill\_discreterainbow}\NormalTok{() }\SpecialCharTok{+}
  \FunctionTok{coord\_flip}\NormalTok{() }\SpecialCharTok{+}
  \FunctionTok{theme}\NormalTok{(}
    \AttributeTok{legend.position =} \StringTok{"none"}\NormalTok{,}
    \AttributeTok{axis.text.x =} \FunctionTok{element\_text}\NormalTok{(}
      \AttributeTok{angle =} \DecValTok{30}\NormalTok{,}
      \AttributeTok{hjust =} \DecValTok{1}
\NormalTok{    )}
\NormalTok{  )}

\FunctionTok{ggsave}\NormalTok{(}
  \AttributeTok{filename =} \FunctionTok{paste0}\NormalTok{(plot\_dir, }\StringTok{"/simple\_time\_to\_max.pdf"}\NormalTok{),}
  \AttributeTok{plot =}\NormalTok{ simple\_time\_to\_max\_plt,}
  \AttributeTok{width =} \DecValTok{8}\NormalTok{,}
  \AttributeTok{height =} \DecValTok{4}
\NormalTok{)}

\NormalTok{simple\_time\_to\_max\_plt}
\end{Highlighting}
\end{Shaded}

\includegraphics{supplemental-material_files/figure-latex/unnamed-chunk-34-1.pdf}

\hypertarget{fitness-in-multi-path-landscape-1}{%
\subsection{Fitness in multi-path landscape}\label{fitness-in-multi-path-landscape-1}}

\begin{Shaded}
\begin{Highlighting}[]
\NormalTok{multipath\_final\_fitness\_plt }\OtherTok{\textless{}{-}} \FunctionTok{ggplot}\NormalTok{(}
    \AttributeTok{data =} \FunctionTok{filter}\NormalTok{(max\_org\_data, landscape }\SpecialCharTok{==} \StringTok{"Multipath"}\NormalTok{),}
    \AttributeTok{mapping =} \FunctionTok{aes}\NormalTok{(}
      \AttributeTok{x =}\NormalTok{ structure,}
      \AttributeTok{y =}\NormalTok{ fitness,}
      \AttributeTok{fill =}\NormalTok{ structure}
\NormalTok{    )}
\NormalTok{  ) }\SpecialCharTok{+}
  \CommentTok{\# geom\_flat\_violin(}
  \CommentTok{\#   position = position\_nudge(x = .2, y = 0),}
  \CommentTok{\#   alpha = .8}
  \CommentTok{\# ) +}
  \FunctionTok{geom\_point}\NormalTok{(}
    \AttributeTok{mapping =} \FunctionTok{aes}\NormalTok{(}\AttributeTok{color =}\NormalTok{ structure),}
    \AttributeTok{position =} \FunctionTok{position\_jitter}\NormalTok{(}\AttributeTok{width =}\NormalTok{ .}\DecValTok{15}\NormalTok{),}
    \AttributeTok{size =}\NormalTok{ .}\DecValTok{5}\NormalTok{,}
    \AttributeTok{alpha =} \FloatTok{0.8}
\NormalTok{  ) }\SpecialCharTok{+}
  \FunctionTok{geom\_boxplot}\NormalTok{(}
    \AttributeTok{width =}\NormalTok{ .}\DecValTok{3}\NormalTok{,}
    \AttributeTok{outlier.shape =} \ConstantTok{NA}\NormalTok{,}
    \AttributeTok{alpha =} \FloatTok{0.5}
\NormalTok{  ) }\SpecialCharTok{+}
  \FunctionTok{scale\_color\_discreterainbow}\NormalTok{() }\SpecialCharTok{+}
  \FunctionTok{scale\_fill\_discreterainbow}\NormalTok{() }\SpecialCharTok{+}
  \FunctionTok{theme}\NormalTok{(}
    \AttributeTok{legend.position =} \StringTok{"none"}\NormalTok{,}
    \AttributeTok{axis.text.x =} \FunctionTok{element\_text}\NormalTok{(}
      \AttributeTok{angle =} \DecValTok{30}\NormalTok{,}
      \AttributeTok{hjust =} \DecValTok{1}
\NormalTok{    )}
\NormalTok{  )}

\FunctionTok{ggsave}\NormalTok{(}
  \AttributeTok{filename =} \FunctionTok{paste0}\NormalTok{(plot\_dir, }\StringTok{"/multipath\_final\_fitness.pdf"}\NormalTok{),}
  \AttributeTok{plot =}\NormalTok{ multipath\_final\_fitness\_plt,}
  \AttributeTok{width =} \DecValTok{6}\NormalTok{,}
  \AttributeTok{height =} \DecValTok{4}
\NormalTok{)}

\NormalTok{multipath\_final\_fitness\_plt}
\end{Highlighting}
\end{Shaded}

\includegraphics{supplemental-material_files/figure-latex/unnamed-chunk-35-1.pdf}

Max fitness over time

\begin{Shaded}
\begin{Highlighting}[]
\NormalTok{multipath\_fitness\_ts\_plt }\OtherTok{\textless{}{-}} \FunctionTok{ggplot}\NormalTok{(}
    \AttributeTok{data =} \FunctionTok{filter}\NormalTok{(max\_org\_data\_ts, landscape }\SpecialCharTok{==} \StringTok{"Multipath"}\NormalTok{),}
    \AttributeTok{mapping =} \FunctionTok{aes}\NormalTok{(}
      \AttributeTok{x =}\NormalTok{ generation,}
      \AttributeTok{y =}\NormalTok{ fitness,}
      \AttributeTok{color =}\NormalTok{ structure,}
      \AttributeTok{fill =}\NormalTok{ structure}
\NormalTok{    )}
\NormalTok{  ) }\SpecialCharTok{+}
  \FunctionTok{stat\_summary}\NormalTok{(}\AttributeTok{fun =} \StringTok{"mean"}\NormalTok{, }\AttributeTok{geom =} \StringTok{"line"}\NormalTok{) }\SpecialCharTok{+}
  \FunctionTok{stat\_summary}\NormalTok{(}
    \AttributeTok{fun.data =} \StringTok{"mean\_cl\_boot"}\NormalTok{,}
    \AttributeTok{fun.args =} \FunctionTok{list}\NormalTok{(}\AttributeTok{conf.int =} \FloatTok{0.95}\NormalTok{),}
    \AttributeTok{geom =} \StringTok{"ribbon"}\NormalTok{,}
    \AttributeTok{alpha =} \FloatTok{0.2}\NormalTok{,}
    \AttributeTok{linetype =} \DecValTok{0}
\NormalTok{  ) }\SpecialCharTok{+}
  \FunctionTok{theme}\NormalTok{(}\AttributeTok{legend.position =} \StringTok{"bottom"}\NormalTok{)}

\FunctionTok{ggsave}\NormalTok{(}
  \AttributeTok{plot =}\NormalTok{ multipath\_fitness\_ts\_plt,}
  \AttributeTok{filename =} \FunctionTok{paste0}\NormalTok{(}
\NormalTok{    plot\_dir,}
    \StringTok{"/multipath\_fitness\_ts.pdf"}
\NormalTok{  ),}
  \AttributeTok{width =} \DecValTok{15}\NormalTok{,}
  \AttributeTok{height =} \DecValTok{10}
\NormalTok{)}

\NormalTok{multipath\_fitness\_ts\_plt}
\end{Highlighting}
\end{Shaded}

\includegraphics{supplemental-material_files/figure-latex/unnamed-chunk-36-1.pdf}

Rank ordering of fitness values

\begin{Shaded}
\begin{Highlighting}[]
\NormalTok{max\_org\_data }\SpecialCharTok{\%\textgreater{}\%}
  \FunctionTok{filter}\NormalTok{(landscape }\SpecialCharTok{==} \StringTok{"Multipath"}\NormalTok{) }\SpecialCharTok{\%\textgreater{}\%}
  \FunctionTok{group\_by}\NormalTok{(structure) }\SpecialCharTok{\%\textgreater{}\%}
  \FunctionTok{summarize}\NormalTok{(}
    \AttributeTok{reps =} \FunctionTok{n}\NormalTok{(),}
    \AttributeTok{median\_fitness =} \FunctionTok{median}\NormalTok{(fitness),}
    \AttributeTok{mean\_fitness =} \FunctionTok{mean}\NormalTok{(fitness)}
\NormalTok{  ) }\SpecialCharTok{\%\textgreater{}\%}
  \FunctionTok{arrange}\NormalTok{(}
    \FunctionTok{desc}\NormalTok{(mean\_fitness)}
\NormalTok{  )}
\end{Highlighting}
\end{Shaded}

\begin{verbatim}
## # A tibble: 7 x 4
##   structure  reps median_fitness mean_fitness
##   <fct>     <int>          <dbl>        <dbl>
## 1 1_3600       50           4.88         4.84
## 2 2_1800       50           4.88         4.78
## 3 3_1200       50           4.74         4.63
## 4 4_900        50           4.64         4.54
## 5 15_240       50           4.06         4.06
## 6 60_60        50           3.94         3.81
## 7 30_120       50           4            3.80
\end{verbatim}

\begin{Shaded}
\begin{Highlighting}[]
\FunctionTok{kruskal.test}\NormalTok{(}
  \AttributeTok{formula =}\NormalTok{ fitness }\SpecialCharTok{\textasciitilde{}}\NormalTok{ structure,}
  \AttributeTok{data =} \FunctionTok{filter}\NormalTok{(max\_org\_data, landscape }\SpecialCharTok{==} \StringTok{"Multipath"}\NormalTok{)}
\NormalTok{)}
\end{Highlighting}
\end{Shaded}

\begin{verbatim}
## 
##  Kruskal-Wallis rank sum test
## 
## data:  fitness by structure
## Kruskal-Wallis chi-squared = 144.73, df = 6, p-value < 2.2e-16
\end{verbatim}

\begin{Shaded}
\begin{Highlighting}[]
\NormalTok{wc\_results }\OtherTok{\textless{}{-}} \FunctionTok{pairwise.wilcox.test}\NormalTok{(}
  \AttributeTok{x =} \FunctionTok{filter}\NormalTok{(max\_org\_data, landscape }\SpecialCharTok{==} \StringTok{"Multipath"}\NormalTok{)}\SpecialCharTok{$}\NormalTok{fitness,}
  \AttributeTok{g =} \FunctionTok{filter}\NormalTok{(max\_org\_data, landscape }\SpecialCharTok{==} \StringTok{"Multipath"}\NormalTok{)}\SpecialCharTok{$}\NormalTok{structure,}
  \AttributeTok{p.adjust.method   =} \StringTok{"holm"}\NormalTok{,}
  \AttributeTok{exact =} \ConstantTok{FALSE}
\NormalTok{)}

\NormalTok{mp\_fitness\_wc\_table }\OtherTok{\textless{}{-}} \FunctionTok{kbl}\NormalTok{(wc\_results}\SpecialCharTok{$}\NormalTok{p.value) }\SpecialCharTok{\%\textgreater{}\%}
  \FunctionTok{kable\_styling}\NormalTok{()}

\FunctionTok{save\_kable}\NormalTok{(}
\NormalTok{  mp\_fitness\_wc\_table,}
  \FunctionTok{paste0}\NormalTok{(plot\_dir, }\StringTok{"/multipath\_fitness\_wc\_table.pdf"}\NormalTok{)}
\NormalTok{)}
\NormalTok{mp\_fitness\_wc\_table}
\end{Highlighting}
\end{Shaded}

\begin{table}
\centering
\begin{tabular}[t]{l|r|r|r|r|r|r}
\hline
  & 1\_3600 & 2\_1800 & 3\_1200 & 4\_900 & 15\_240 & 30\_120\\
\hline
2\_1800 & 1.0000000 & NA & NA & NA & NA & NA\\
\hline
3\_1200 & 0.0389539 & 0.2309342 & NA & NA & NA & NA\\
\hline
4\_900 & 0.0000552 & 0.0022081 & 0.6036094 & NA & NA & NA\\
\hline
15\_240 & 0.0000000 & 0.0000001 & 0.0000387 & 0.0022081 & NA & NA\\
\hline
30\_120 & 0.0000000 & 0.0000000 & 0.0000000 & 0.0000003 & 0.4456978 & NA\\
\hline
60\_60 & 0.0000000 & 0.0000000 & 0.0000002 & 0.0000094 & 0.6036094 & 1\\
\hline
\end{tabular}
\end{table}

\hypertarget{valleys-crossed-in-valley-crossing-landscape-1}{%
\subsection{Valleys crossed in valley-crossing landscape}\label{valleys-crossed-in-valley-crossing-landscape-1}}

\begin{Shaded}
\begin{Highlighting}[]
\NormalTok{valleycrossing\_valleys\_plt }\OtherTok{\textless{}{-}} \FunctionTok{ggplot}\NormalTok{(}
    \AttributeTok{data =} \FunctionTok{filter}\NormalTok{(max\_org\_data, landscape }\SpecialCharTok{==} \StringTok{"Valley crossing"}\NormalTok{),}
    \AttributeTok{mapping =} \FunctionTok{aes}\NormalTok{(}
      \AttributeTok{x =}\NormalTok{ structure,}
      \AttributeTok{y =}\NormalTok{ valleys\_crossed,}
      \AttributeTok{fill =}\NormalTok{ structure}
\NormalTok{    )}
\NormalTok{  ) }\SpecialCharTok{+}
  \CommentTok{\# geom\_flat\_violin(}
  \CommentTok{\#   position = position\_nudge(x = .2, y = 0),}
  \CommentTok{\#   alpha = .8}
  \CommentTok{\# ) +}
  \FunctionTok{geom\_point}\NormalTok{(}
    \AttributeTok{mapping =} \FunctionTok{aes}\NormalTok{(}\AttributeTok{color =}\NormalTok{ structure),}
    \AttributeTok{position =} \FunctionTok{position\_jitter}\NormalTok{(}\AttributeTok{width =}\NormalTok{ .}\DecValTok{15}\NormalTok{),}
    \AttributeTok{size =}\NormalTok{ .}\DecValTok{5}\NormalTok{,}
    \AttributeTok{alpha =} \FloatTok{0.8}
\NormalTok{  ) }\SpecialCharTok{+}
  \FunctionTok{geom\_boxplot}\NormalTok{(}
    \AttributeTok{width =}\NormalTok{ .}\DecValTok{3}\NormalTok{,}
    \AttributeTok{outlier.shape =} \ConstantTok{NA}\NormalTok{,}
    \AttributeTok{alpha =} \FloatTok{0.5}
\NormalTok{  ) }\SpecialCharTok{+}
  \FunctionTok{scale\_color\_discreterainbow}\NormalTok{() }\SpecialCharTok{+}
  \FunctionTok{scale\_fill\_discreterainbow}\NormalTok{() }\SpecialCharTok{+}
  \FunctionTok{theme}\NormalTok{(}
    \AttributeTok{legend.position =} \StringTok{"none"}\NormalTok{,}
    \AttributeTok{axis.text.x =} \FunctionTok{element\_text}\NormalTok{(}
      \AttributeTok{angle =} \DecValTok{30}\NormalTok{,}
      \AttributeTok{hjust =} \DecValTok{1}
\NormalTok{    )}
\NormalTok{  )}
\FunctionTok{ggsave}\NormalTok{(}
  \AttributeTok{filename =} \FunctionTok{paste0}\NormalTok{(plot\_dir, }\StringTok{"/valleycrossing\_valleys\_crossed.pdf"}\NormalTok{),}
  \AttributeTok{plot =}\NormalTok{ valleycrossing\_valleys\_plt,}
  \AttributeTok{width =} \DecValTok{6}\NormalTok{,}
  \AttributeTok{height =} \DecValTok{4}
\NormalTok{)}

\NormalTok{valleycrossing\_valleys\_plt}
\end{Highlighting}
\end{Shaded}

\includegraphics{supplemental-material_files/figure-latex/unnamed-chunk-39-1.pdf}

\begin{Shaded}
\begin{Highlighting}[]
\NormalTok{vc }\OtherTok{\textless{}{-}}\NormalTok{ max\_org\_data }\SpecialCharTok{\%\textgreater{}\%}
  \FunctionTok{filter}\NormalTok{(landscape }\SpecialCharTok{==} \StringTok{"Valley crossing"}\NormalTok{) }\SpecialCharTok{\%\textgreater{}\%}
  \FunctionTok{group\_by}\NormalTok{(structure) }\SpecialCharTok{\%\textgreater{}\%}
  \FunctionTok{summarize}\NormalTok{(}
    \AttributeTok{reps =} \FunctionTok{n}\NormalTok{(),}
    \AttributeTok{median\_valleys\_crossed =} \FunctionTok{median}\NormalTok{(valleys\_crossed),}
    \AttributeTok{mean\_valleys\_crossed =} \FunctionTok{mean}\NormalTok{(valleys\_crossed),}
    \AttributeTok{min\_valleys\_crossed =} \FunctionTok{min}\NormalTok{(valleys\_crossed)}
\NormalTok{  ) }\SpecialCharTok{\%\textgreater{}\%}
  \FunctionTok{arrange}\NormalTok{(}
    \FunctionTok{desc}\NormalTok{(mean\_valleys\_crossed)}
\NormalTok{  )}
\NormalTok{vc}
\end{Highlighting}
\end{Shaded}

\begin{verbatim}
## # A tibble: 7 x 5
##   structure  reps median_valleys_crossed mean_valleys_crossed
##   <fct>     <int>                  <dbl>                <dbl>
## 1 1_3600       50                  100                  100  
## 2 2_1800       50                  100                  100  
## 3 3_1200       50                  100                   99.6
## 4 4_900        50                   89.5                 89.3
## 5 30_120       50                   47                   47.2
## 6 15_240       50                   46                   46.6
## 7 60_60        50                   46                   45.5
## # i 1 more variable: min_valleys_crossed <dbl>
\end{verbatim}

\begin{Shaded}
\begin{Highlighting}[]
\NormalTok{vc}\SpecialCharTok{$}\NormalTok{min\_valleys\_crossed}
\end{Highlighting}
\end{Shaded}

\begin{verbatim}
## [1] 100 100  89  76  31  32  32
\end{verbatim}

\begin{Shaded}
\begin{Highlighting}[]
\FunctionTok{kruskal.test}\NormalTok{(}
  \AttributeTok{formula =}\NormalTok{ valleys\_crossed }\SpecialCharTok{\textasciitilde{}}\NormalTok{ structure,}
  \AttributeTok{data =} \FunctionTok{filter}\NormalTok{(max\_org\_data, landscape }\SpecialCharTok{==} \StringTok{"Valley crossing"}\NormalTok{)}
\NormalTok{)}
\end{Highlighting}
\end{Shaded}

\begin{verbatim}
## 
##  Kruskal-Wallis rank sum test
## 
## data:  valleys_crossed by structure
## Kruskal-Wallis chi-squared = 309.49, df = 6, p-value < 2.2e-16
\end{verbatim}

\begin{Shaded}
\begin{Highlighting}[]
\NormalTok{wc\_results }\OtherTok{\textless{}{-}} \FunctionTok{pairwise.wilcox.test}\NormalTok{(}
  \AttributeTok{x =} \FunctionTok{filter}\NormalTok{(max\_org\_data, landscape }\SpecialCharTok{==} \StringTok{"Valley crossing"}\NormalTok{)}\SpecialCharTok{$}\NormalTok{valleys\_crossed,}
  \AttributeTok{g =} \FunctionTok{filter}\NormalTok{(max\_org\_data, landscape }\SpecialCharTok{==} \StringTok{"Valley crossing"}\NormalTok{)}\SpecialCharTok{$}\NormalTok{structure,}
  \AttributeTok{p.adjust.method   =} \StringTok{"holm"}\NormalTok{,}
  \AttributeTok{exact =} \ConstantTok{FALSE}
\NormalTok{)}

\NormalTok{vc\_valleys\_crossed\_wc\_table }\OtherTok{\textless{}{-}} \FunctionTok{kbl}\NormalTok{(wc\_results}\SpecialCharTok{$}\NormalTok{p.value) }\SpecialCharTok{\%\textgreater{}\%}
  \FunctionTok{kable\_styling}\NormalTok{()}

\FunctionTok{save\_kable}\NormalTok{(}
\NormalTok{  vc\_valleys\_crossed\_wc\_table,}
  \FunctionTok{paste0}\NormalTok{(plot\_dir, }\StringTok{"/valley\_crossing\_valleys\_wc\_table.pdf"}\NormalTok{)}
\NormalTok{)}
\NormalTok{vc\_valleys\_crossed\_wc\_table}
\end{Highlighting}
\end{Shaded}

\begin{table}
\centering
\begin{tabular}[t]{l|r|r|r|r|r|r}
\hline
  & 1\_3600 & 2\_1800 & 3\_1200 & 4\_900 & 15\_240 & 30\_120\\
\hline
2\_1800 & NaN & NA & NA & NA & NA & NA\\
\hline
3\_1200 & 0.796952 & 0.796952 & NA & NA & NA & NA\\
\hline
4\_900 & 0.000000 & 0.000000 & 0 & NA & NA & NA\\
\hline
15\_240 & 0.000000 & 0.000000 & 0 & 0 & NA & NA\\
\hline
30\_120 & 0.000000 & 0.000000 & 0 & 0 & 0.9787605 & NA\\
\hline
60\_60 & 0.000000 & 0.000000 & 0 & 0 & 0.9787605 & 0.796952\\
\hline
\end{tabular}
\end{table}

  \bibliography{packages.bib,supplemental.bib}

\end{document}
